    \documentclass[12pt]{article}
    \usepackage[utf8]{inputenc}
    \usepackage{polski}
    \usepackage{enumitem}
    \usepackage{amsmath}


    \title{Zadanie 1--1}
    \author{Patryk Lisik}
    \date{\(6\) Lutego  2023}

    \begin{document}

    \maketitle
    \renewcommand{\abstractname}{Treść}

    \begin{abstract}
    \end{abstract}


    \section*{Rozwiązanie}

    Entropia źródła
    \begin{equation*}
        H(X) = -\sum_i p_i \log p_i = -(0.2 \log 0.2 + 0.8 \log 0.8) = 0.72
    \end{equation*}

    Informacja wzajemna
    \begin{multline*}
        I(X,Y)  = \sum_{i}\sum_j P(x_j,y_j)\log_2 \frac{P(x_i|y_j)}{P(x_i)}
        = \sum_{i}\sum_j P(y_j|x_i)P(x_i)\log_2 \frac{P(x_i|y_j)}{P(x_i)} \\
        = P(y_0|x_0)P(x_0)\log_2 \frac{P(x_0|y_0)}{P(x_0)} \\
        + P(y_1|x_0)P(x_0)\log_2 \frac{P(x_0|y_1)}{P(x_0)} \\
        + P(y_0|x_1)P(x_1)\log_2 \frac{P(x_1|y_0)}{P(x_1)} \\
        + P(y_1|x_1)P(x_1)\log_2 \frac{P(x_1|y_1)}{P(x_1)} \\
    \end{multline*}
Pojemność binarnego kanału symerycznego
    \begin{multline*}
    C_s = \Omega(\alpha + p - 2\alpha p) - \Omega (p) = \Omega(0.2+0.25 - 2 \cdot 0.2 \cdot 0.25)-\Omega(0.25) \\
        = \Omega(0.35)-\Omega(0.25) \\
        = 0.35 \log_2 0.35 - (1-0.35) \log_2(1-0.35) -(-0.25\log_2 0.25 - (1-0.25)\log_2(1-0.25) \approx 0.12 
\end{multline*}
    \end{document}
