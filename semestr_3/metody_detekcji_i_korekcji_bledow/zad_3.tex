
    \documentclass[12pt]{article}
    \usepackage[utf8]{inputenc}
    \usepackage{polski}
    \usepackage{enumitem}
    \usepackage{amsmath}

    \newcommand{\floor}[1]{\left\lfloor #1 \right\rfloor}
    \setcounter{MaxMatrixCols}{20}

    \title{Zadanie 3}
    \author{Patryk Lisik}
    \date{\(10\) Lutego  2024}

    \begin{document}

    \maketitle
    \renewcommand{\abstractname}{Treść}

    \begin{abstract}
    Binarny liniowy kod cykliczny $C_{cyc}(n,k)$ o długości kodu $n=14$ ma wielomian generujący
        $g(x) = 1 \oplus x^2 \oplus x^6$.

    \begin{enumerate}[label=(\alph*)]
        \item Wyznacz liczbę bitów wiadomości $k$ i przystości w każdym słowie kodowym $\mathbf{c}$
        \item Wyznacz liczbę słów kodowych kodu $M$.
        \item Wyznacz macierz generującą $\mathbf{G}$ i kontroli przystości $\mathbf{H}$ kodu. 
        \item Wyznacz minimalną odległość Hamminga $d_{min}$ kodu. 
    \end{enumerate}
    \end{abstract}



    \section*{Rozwiązanie}
    \subsection*{a) Wyznacz liczbę bitów wiadomości $k$ i przystości w każdym słowie kodowym $\mathbf{c}$} 
    Niech $r$ będzie stopniem wielominu generującego $g(x)$. Jeśli wielomian jest stopnia $r$ to genruje kod cykliczny
    $C_{cyc}(n,l) $ taki, że $r=n-k$. Wielomin $g(x)$ jest stopnia 6, co onzacza że kod jest $C_{cyc}(14, 8)$.   
    Kod ma 8 bitów wiadomości o 6 bitów parzystości.

    \subsection*{b) Wyznacz liczbę słów kodowych kodu $M$} 

    $$M=2^k=2^8=256 $$
    \subsection*{c) Wyznacz macierz generującą $\mathbf{G}$ i kontroli przystości $\mathbf{H}$ kodu} 
    Liniowy kod cykliczny jest rozpianany przez $k=8$ wielomianów kodowych $g(x),xg(x),\dots x^{6}g(x)$. 
    Oznacza to, że macierz $G$ skłąda się z przesuniętych wektorów 
    $ ( \underbrace{1}_{x^0}
      \underbrace{0}_{x^1}
      \underbrace{1}_{x^2}
      \underbrace{0}_{x^3}
      \underbrace{0}_{x^5}
      \underbrace{1}_{x^6}
      \underbrace{0}_{x^7}
      \underbrace{0}_{x^8}
      \underbrace{0}_{x^9}
      \underbrace{0}_{x^{10}}
      \underbrace{0}_{x^{11}}
      \underbrace{0}_{x^{12}}
      \underbrace{0}_{x^{13}}
      \underbrace{0}_{x^{14}}
    )
    $ 
    \begin{equation*}
        \mathbf{G} = \begin{pmatrix}
           1 & 0 & 1 & 0 & 0 & 1 & 0 & 0 & 0 & 0 & 0 & 0 & 0 & 0  \\
           0 & 1 & 0 & 1 & 0 & 0 & 1 & 0 & 0 & 0 & 0 & 0 & 0 & 0  \\ 
           0 & 0 & 1 & 0 & 1 & 0 & 0 & 1 & 0 & 0 & 0 & 0 & 0 & 0  \\ 
           0 & 0 & 0 & 1 & 0 & 1 & 0 & 0 & 1 & 0 & 0 & 0 & 0 & 0  \\ 
           0 & 0 & 0 & 0 & 1 & 0 & 1 & 0 & 0 & 1 & 0 & 0 & 0 & 0  \\ 
           0 & 0 & 0 & 0 & 0 & 1 & 0 & 1 & 0 & 0 & 1 & 0 & 0 & 0  \\ 
           0 & 0 & 0 & 0 & 0 & 0 & 1 & 0 & 1 & 0 & 0 & 1 & 0 & 0  \\ 
           0 & 0 & 0 & 0 & 0 & 0 & 0 & 1 & 0 & 1 & 0 & 0 & 1 & 0  \\ 
           0 & 0 & 0 & 0 & 0 & 0 & 0 & 0 & 1 & 0 & 1 & 0 & 0 & 1 
       \end{pmatrix} 
    \end{equation*}

    Macierz kontroli parzystości
    \begin{equation*}
        \mathbf{H} = (I_{n-k} \quad P^T ) = \begin{pmatrix}
            1 & 0 & 0 & 0 & 0 & 0 &  \quad  1 & 0 & 0 & 0 & 1 & 0 & 1 & 0  \\
            0 & 1 & 0 & 0 & 0 & 0 &  \quad  0 & 1 & 0 & 0 & 0 & 1 & 0 & 1  \\
            0 & 0 & 1 & 0 & 0 & 0 &  \quad  1 & 0 & 1 & 0 & 1 & 0 & 0 & 0  \\
            0 & 0 & 0 & 1 & 0 & 0 &  \quad  0 & 1 & 0 & 1 & 0 & 1 & 0 & 0  \\
            0 & 0 & 0 & 0 & 1 & 0 &  \quad  0 & 0 & 1 & 0 & 1 & 0 & 1 & 0  \\
            0 & 0 & 0 & 0 & 0 & 1 &  \quad  0 & 0 & 0 & 1 & 0 & 1 & 0 & 1  \\
        \end{pmatrix}
    \end{equation*}
    \subsection*{d) Wyznacz minimalną odległość Hamminga $d_{min}$ kodu}

    Dla linowego kodu blokowego, odległość minimalna kodu jest równa minimalnej liczbie kolumn z macierzy $\mathbf{H}$,
    które dodane razem dają $0$. 

    \begin{table}[h]
        \centering
        \begin{tabular}{ccccccccccccccc}
            $c_0$ &  $c_1$ & $c_2$ & $c_3$ &  $c_4$ & $c_5$ & $c_6$ & $c_7$ & $c_8$ & $c_9$ &$c_{10}$ & $c_{11}$ & $c_{12}$ & $c_{13}$  \\ \hline
            1     &    0   & 0     & 0     & 0      & 0     & 1     & 0     & 0     & 0     & 1      & 0      & 1      & 0       \\
            0     &    1   & 0     & 0     & 0      & 0     & 0     & 1     & 0     & 0     & 0      & 1      & 0      & 1       \\
            0     &    0   & 1     & 0     & 0      & 0     & 1     & 0     & 1     & 0     & 1      & 0      & 0      & 0       \\
            0     &    0   & 0     & 1     & 0      & 0     & 0     & 1     & 0     & 1     & 0      & 1      & 0      & 0       \\
            0     &    0   & 0     & 0     & 1      & 0     & 0     & 0     & 1     & 0     & 1      & 0      & 1      & 0       \\
            0     &    0   & 0     & 0     & 0      & 1     & 0     & 0     & 0     & 1     & 0      & 1      & 0      & 1       \\
        \end{tabular}
    \end{table}

    $$d_{min}=3 $$







    \end{document}
