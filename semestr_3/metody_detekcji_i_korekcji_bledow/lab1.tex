\documentclass[12pt]{article}
\usepackage[utf8]{inputenc}
\usepackage{polski}
\usepackage{enumitem}
\usepackage{amsmath}


\title{Teoria informacji -- Lab1}
\author{Patryk Lisik}
\date{\(19\) Listopad  2023}

\begin{document}

\maketitle


\section*{Problem 1}

Dane jest niepamiętające dyskretne źródło informacji z alfabetem $A = \{a,b,c,d\}$ i prawdopodobieństwem  
nadania każdego znaku odpowiednio $P(X=a)=\frac{1}{2}, P(X=b)=P(X=c)=\frac{1}{8}, P(X=d) = \frac{1}{2}$. 
Jaka jest entropia źródła?

\begin{table}[h]
\begin{tabular}{|l|l|l|l|l|}
\hline
x\_i & a   & b   & c   & d   \\ \hline
p\_i & 1/2 & 1/8 & 1/8 & 1/4 \\ \hline
I\_i & 1b  & 3b  & 3b  & 2b  \\ \hline
\end{tabular}

\end{table}
Ilość informacji $I_i = \log_2\frac{1}{p_i}$

$$H(X) = \sum p_iI_i = 1.75b $$

\section*{Problem 2}
Źródło o szerokości pasma $W = 4000Hz$ zostało poddane próbkowaniu z częstotliwością Nyquista.
Przyjmując 


\begin{table}[h]
\begin{tabular}{|l|l|l|l|l|l|}
\hline
    x\_i & -2   & -1   & 0   & 1 & 2   \\ \hline
    p\_i & 1/2 & 1/4 & 1/8 & 1/16 & 1/16 \\ \hline
    I\_i & 1b  & 3b  & 3b  & 2b & 2b \\ \hline
\end{tabular}

\end{table}

$$H(X) = 1.875b$$

Musimy próbkowań z częstotliwością dwukrotnie większą niż źródło = $8000$ hz

$$8000 \cdot 1.875b = 15000 b/s $$

\section*{Problem 3}
Oblicz tempo informacji źródła nadającego r = 3000 znaków na sekundę z zakresu czterech znaków z prawdopodobieństwami danymi w tabeli.

\begin{table}[h]
\begin{tabular}{l|lllll}
\hline \hline
$x_i$ & A   & B   & C   & C   \\ \hline
$P_i$ & $\frac{1}{3}$ & $\frac{1}{3}$ &$\frac{1}{6}$& $\frac{1}{6}$\\ 
$I_i(bit)$ & $\log_23$  & $\log_23$  & $1+\log_23$  & $1+\log_23$ \\  \hline \hline
\end{tabular}

\end{table}
$$\log_26 = \log_2(2\cdot 3)=\log_2 2 + \log_23 = 1 + \log_23 $$
$$\log_23 \approx 1.585 $$
$$ H(X) =\frac{1}{3}\log_23 + \frac{1}{3}\log_23 + \frac{1}{6}\log_26 \frac{1}{6}\log_26 =\frac{1}{3} + \log_23 \approx = 1.918b $$

\section*{Problem 4}

Zbuduj rozszerzenie 2. rzędu źródła z Problemu 1 i oblicz jego entropię.

\begin{table}[h]
\hspace*{-1.5cm}
\begin{tabular}{|l|l|l|l|l|l|l|l|l|l|l|l|l|l|l|l|l|}
\hline
$y_1=x_j x_k$ & aa  & ab   & ac   & ad  & ba   & bb   & bc   & bd   & ca   & cb   & cc   & cd   & da  & db   & dc   & dd   \\ \hline
    $p_1$        & $\frac{1}{4}$ & $\frac{1}{16}$ & $\frac{1}{16}$ & $\frac{1}{8}$ & $\frac{1}{16}$ & $\frac{1}{64}$ & $\frac{1}{64}$ & $\frac{1}{32}$ & $\frac{1}{16}$ & $\frac{1}{64}$ & $\frac{1}{64}$ & $\frac{1}{32}$ & $\frac{1}{8}$ & $\frac{1}{32}$ & $\frac{1}{32}$ & $\frac{1}{16}$ \\[0.4em] \hline
$I_i $     & 2b  & 4b   & 4b   & 3b  & 4b   & 6b   & 6b   & 5b   & 4b   & 6b   & 6b   & 5b   & 3b  & 5b   & 5b   & 4b   \\ \hline
\end{tabular}
\caption{Rozszeżenie żródła z problmu 1}
\end{table}

$$H(X^2) = \frac{1}{4}\cdot 2b+\frac{2}{8}\cdot 3b + \frac{5}{16}\cdot 4b + \frac{4}{32} \cdot 5b + 
\frac{4}{64} 6b =3.5b = 2H(X)  $$

\section*{Problem 5}
Rozważ kanał binarny (niesymetryczny) o macierzy kanału
$$ \mathbf{P}= 
\begin{pmatrix}
    \frac{3}{4} & \frac{1}{4} \\
    \frac{1}{8} & \frac{7}{8} 
\end{pmatrix}
$$
o prawdopodobieństwach wejścia $P(X = 0) = \frac{4}{5}, P(X = 1)=\frac{1}{5}$. 
Oblicz prawdopodobieństwo wyjściowe i prawdopodobieństwo wstecz. 

$$p=\left( \frac{4}{5} \frac{1}{5}  \right) $$
$$H(X) = \frac{4}{5}\log_2(\frac{5}{4}) + \frac{1}{5} \log_25 = \log_25 - \frac{8}{5}  $$

Prawdopodobieństwa wyjścia wyjścia:

$$q = p\mathbf{P} = (\frac{4}{5} \frac{1}{5} )   
\begin{pmatrix}
    \frac{3}{4} & \frac{1}{4} \\
    \frac{1}{8} & \frac{7}{8} 
\end{pmatrix}
= \left( \frac{25}{40} \frac{15}{40}  \right) 
= \left( \frac{5}{8} \frac{3}{8}  \right) 
$$

Entropia wyjściowa:

$$H(Y) = \frac{5}{8}\log_2\frac{8}{5} + \frac{3}{8}\log_2\frac{8}{3} = 0.954b $$

Macierz prawdopodobieństw włączanych 

$$ R =
\begin{pmatrix}
    \frac{4}{5} & 0 \\
    0 & \frac{1}{5} 
\end{pmatrix}
\begin{pmatrix}
    \frac{3}{4} & \frac{1}{4} \\
    \frac{1}{8} & \frac{7}{8} 
\end{pmatrix}=
\begin{pmatrix}
    \frac{3}{5} & \frac{1}{5} \\
    \frac{1}{40} & \frac{7}{40} 
\end{pmatrix}
$$
Entropia łączna
Wynik powinien sumować się do 1. 

\begin{multline*}
    H(X,Y)= \frac{3}{5}\log_2\frac{5}{3} + \frac{1}{5}\log_25 + \frac{1}{40}\log_2 40 + \frac{7}{40}\frac{40}{7}\\
    = \frac{5}{3}\log_25 - \frac{3}{5}\log_23+\frac{1}{5}\log25+\frac{1}{40}\log25+\frac{3}{40}+\frac{7}{40}\log25+\frac{21}{40}
    -\frac{7}{40}\log_27 \\
    = \frac{3}{5} + \log25 - \frac{3}{5}\log_23 - \frac{7}{40}\log_27 \approx 1.780 b
\end{multline*}

Entropia szumu :
\begin{multline*}
    H(Y|X) = \frac{3}{5}\log_2 \frac{4}{3} + \frac{1}{5}\log_24 + \frac{1}{40}\log_2{8} + \frac{7}{40}\log_2 \frac{7}{8}=\\
    = \frac{6}{5} - \frac{3}{5}\log_23 + \frac{2}{5} + \frac{3}{40} + \frac{21}{40} - \frac{7}{40}\log_27 = 
    \frac{11}{5} - \frac{3}{5}\log_23 - \frac{7}{40}\log_2 7
\end{multline*}

Macierz Q / Macierz wstecz

$$Q = 
\begin{pmatrix}
    \frac{3}{5} & \frac{1}{5} \\
    \frac{1}{40} & \frac{7}{40} 
\end{pmatrix}
\begin{pmatrix}
    \frac{8}{5} & 0 \\
    0 & \frac{8}{3} 
\end{pmatrix}=
\begin{pmatrix}
    \frac{24}{25} & \frac{8}{15} \\
    \frac{1}{25} & \frac{7}{15} 
\end{pmatrix}
$$

Ważne 

Tu nie ma mnożenie macierzowego 
$$p_iP_{ij} =R_{ij} = Q_{ij}q_j  $$
$$q_j = \sum_i p_iP_{ij} $$

\section*{Problem 6}
Wyznacz entropię \textit{a priori} i \textit{a posteriori} dla poprzedniego kanału.

Ekwiwokacja
$$H(X|Y) = \frac{3}{5}\log \frac{25}{24} + \frac{1}{5}\log{15}{8} + \frac{1}{40}\log_2 25 + \frac{7}{40}\log_2\frac{15}{7}
= \frac{6}{5}\log_2 5 - \frac{3}{5}\log_23 - \frac{9}{5}+\frac{1}{5}\log_23+\frac{1}{20}\log_2{5}+\frac{7}{40}\log_25+\frac{7}{40}\log_23-\frac{7}{40}\log_27
=\frac{65}{40}\log_25 - \frac{12}{5} - \frac{9}{40}\log_23 - \frac{7}{40}\log_27 \approx 0.525b
$$

Informacja wzajemna
$$ I(X,Y) = H(X)+ H(Y) - H(X,Y) = \frac{4}{5} + \frac{9}{40}\log_23 - \frac{5}{8}\log_25 + \frac{7}{40}\log_27 \approx 0.197b $$

\section*{Problem 7}
Wyznacz pojemność poniższego kanału: 

$$
\mathbf{P}=
\begin{pmatrix}
    \frac{1}{2} & \frac{1}{4} & \frac{1}{4} \\
    \frac{1}{4} & \frac{1}{2} & \frac{1}{4} \\
    \frac{1}{4} & \frac{1}{4} & \frac{1}{2} \\
\end{pmatrix}
$$

$$p = (x\quad y \quad z ) $$
$$ x+y+z = 1 $$
$$ C = \max I(X,Y) = \max [H(X) + H(Y) - H(X,Y) ] $$ 
$$I(X;Y) = H(X) + H(Y) - H(X,Y) $$

\begin{multline*}
    q=(x \quad y \quad z ) 
    \begin{pmatrix}
        \frac{1}{2} & \frac{1}{4} & \frac{1}{4} \\
        \frac{1}{4} & \frac{1}{2} & \frac{1}{4} \\
        \frac{1}{4} & \frac{1}{4} & \frac{1}{2} \\
    \end{pmatrix}  \\
    = \left( \frac{2x+y+z}{4} \quad \frac{x+2y+z}{4} \quad \frac{x + y +2z}{4}  \right) \\
    = \left( \frac{1+x}{4} \quad \frac{1+y}{4} \quad \frac{1+z}{4} \right)
\end{multline*}


$$ H(X) = -x \log_2 X - y\log y - z \log z $$ 

\begin{multline*}
H(Y) =  -\frac{2x+y+z}{4}\log_2 \frac{2x+y+z}{4} - \frac{x+2y+z}{4}\log_2 \frac{x+2y+z}{4} - \\
    \frac{x + y +2z}{4}\log_2 \frac{x + y +2z}{4} = \\
    = 2 - \frac{1}{4}\cdot\left[(1+x)\log_2(1+x) + (1+y)]\log_2(1+y)+(1+z)\log_2(1+z) \right]
\end{multline*}
$$
R =\overline{p}P=
\begin{pmatrix}
    x &    &  \\
      & y  &  \\
      &    & z \\
\end{pmatrix}
\begin{pmatrix}
    \frac{1}{2} & \frac{1}{4} & \frac{1}{4} \\
    \frac{1}{4} & \frac{1}{2} & \frac{1}{4} \\
    \frac{1}{4} & \frac{1}{4} & \frac{1}{2} \\
\end{pmatrix}
= 
\begin{pmatrix}
    \frac{x}{2} & \frac{x}{4} & \frac{x}{4} \\
    \frac{y}{4} & \frac{y}{2} & \frac{y}{4} \\
    \frac{z}{4} & \frac{z}{4} & \frac{z}{2} \\
\end{pmatrix}
$$

\begin{multline*}
H(X,Y) =  -\frac{x}{2}\log_2\frac{x}{2}-\frac{x}{4}\log_2\frac{x}{4} - \\
 -\frac{y}{2}\log_2\frac{y}{2}-\frac{y}{4}\log_2\frac{y}{4} = \dots = 
    = \frac{3}{2}-x\log_2x-y\log_2y-z\log_2z
\end{multline*}

\begin{multline*}
    I(X,Y) = \frac{1}{2}- \frac{1}{4} [(1+x)\log_2(1+x)+(1+y)\log_2(1+y)+(1+z)\log_2(1+z)]
\end{multline*}

$$F(x,y,z) = \frac{1}{2}- \frac{1}{4} [(1+x)\log_2(1+x)+(1+y)\log_2(1+y)+(1+z)\log_2(1+z)] - \lambda (x+y+z-1)$$

Metoda Lagranga 

\begin{align*}
    \begin{cases}
        \frac{\partial F}{x} = \frac{1}{4}[\log_2(1+x) + \frac{1}{\ln2}] + \lambda = 0 \\
        \frac{\partial F}{y} = \frac{1}{4}[\log_2(1+y) + \frac{1}{\ln2}] + \lambda = 0 \\
        \frac{\partial F}{z} = \frac{1}{4}[\log_2(1+z) + \frac{1}{\ln2}] + \lambda = 0 \\
        G(x,y,z) = x+y+z - 1 = 0
    \end{cases}
\end{align*}

$$ \log_2(1+x) = 4\lambda - \ln 2 $$
$$ 1+x = \frac{2^{4\lambda}}{2^{\ln 2}} $$

\begin{align*}
    x = \frac{2^{4\lambda}}{2^{\ln 2}} - 1 \\
    y = \frac{2^{4\lambda}}{2^{\ln 2}} - 1 \\
    z = \frac{2^{4\lambda}}{2^{\ln 2}} - 1 \\
\end{align*}

\begin{align*}
    x= \frac{1}{3}
    y= \frac{1}{3}
    z= \frac{1}{3}
\end{align*}

\section*{Lab 2}

\subsection*{Problem 1}

$$GF(2) = \mathcal{F} = < \{0,1 \}>, \oplus \cdot $$

\begin{table}[h]
\begin{tabular}{l|ll}
$\oplus$  & 0 & 1 \\ \hline
0 & 0 & 1 \\
1 & 1 & 1
\end{tabular}
\end{table}

\begin{table}[h]
\begin{tabular}{l|ll}
$\cdot$  & 0 & 1 \\ \hline
0 & 0 & 0 \\
1 & 0 & 1
\end{tabular}
\end{table}


\begin{align*}
    e_1 = (1000)
    e_2 = (0100)
    e_3 = (0010)
    e_4 = (0001)
\end{align*}

$$v=a_1e_1 + a_2e_2+a_3e_3+a_4e_4 a_i \quad \in GF(2)$$
$$v=(a_1 \quad a_2 \quad a_3 \quad a_4) \quad  a_i \in GF(2)$$

$$
V_4 = 
\begin{Bmatrix}
    v_0 = (0000) & v_4 = (0010) & v_8 = (0001)    & v_{12} = (0011)   \\
    v_1 = (1000) & v_5 = (1010) & v_9 = (0011)    & v_{13} = (1011)   \\
    v_2 = (0100) & v_6 = (0110) & v_{10} = (0101) & v_{14} = (0111)\\
    v_3 = (1100) & v_7 = (1110) & v_{11} = (1101) & v_{15} = (1111)\\
\end{Bmatrix}
$$

$$v=(a_1a_2a_3a_4 $$
$$w=(b_1b_2b_3b_4 $$

Definicje działań

$$v\oplus w  = (a_1 \oplus b_1 \quad a_2 \oplus b_2 \quad a_3 \oplus b_3 \quad a_4 \oplus b_4 \quad$$
$$aV = (aa_1 \quad  aa_1 \quad aa_3 \quad aa_3 \quad aa_4 \quad$$ 

$$ \forall_{c \ in v_4} 0(a_1a_2a_3a_4) = (0000) $$

$$ \forall_{c \ in v_4} 1(a_1a_2a_3a_4) = (a_1a_2a_3a_4) $$
$$ \forall_{c \ in v_4} v\oplus c = 0 $$

$$ S= {(0000),(1001), (0100),(1101)  = {v_1,v_9,v_2,v_{11}$$

$v_x$ odpowida char w C


\begin{table}[h]
\begin{tabular}{llll}
$v_0 \oplus v_0 = 0$   & &                         &  \\ 
$v_0 \oplus v_9 = v_9$ & $v_9 \oplus v_9 = v_0 $ &  &\\
$v_0 \oplus v_2 = v_2$ & $v_0 \oplus v_2 = v_11$ & $v_2 \oplus v_2 = 0$ & \\
    $v_0 \oplus v_{11} = v_{11}$ & $v_9 \oplus v_11 = v_2$ & $v_2 \oplus v_{11} = v_9 $ & $v_{11} \oplus v_{11} = 0 $ \\
\end{tabular}
\end{table}


$$f_1 = (1001) \quad f_2 =(0100) \quad u = a_1f_1+a_2f_2 \quad a_1 \in GF(2) $$

\begin{align*}
    u_0 = 0f_1+0f_2 = (0000) = v_0 \\
    u_1 = 1f_1+0f_2 = f_1  = (1001) = v_9 \\
    u_2 = 0f_1+1f_2 = f_2 = (0100) = v_2 \\
    u_3 = 1f_1+1f_2 = (1101) = v_{11} \\
\end{align*}

$$<(1001),(0100)> = \{(0000), (1001),(0100),(1101)\} = S $$

\subsection*{Chyba nowe zadanie???}

$$S \subset V_4 $$
$$S = {v_0, v_12, v_6, v_2, v_10, v_14, v_4, v_8} \quad S^{\perp} = ?$$

$$S^{\perp} = \{ w \in V_4: \quad \forall_{V \in S} wv=0 \} $$

Dokończyć jak ktoś wstawi zdjęcie

\subsection*{To z macieżą G}

Deklarujmy że można perzekszałcić macież $G$ w $G'$ za pomocą działań elementarnych. 
Czyli zamiana wierszy, kombinacja liniowa wierszy. 

$$
G=
\begin{pmatrix}
    1 & 0 & 1 & 1 & 0 \\
    0 & 1 & 0 & 0 & 1 \\
    1 & 1 & 0 & 1 & 1 \\
\end{pmatrix}
\quad
G'=
\begin{pmatrix}
    0 & 1 & 0 & 0 & 1 \\
    1 & 0 & 1 & 1 & 0 \\
    1 & 0 & 0 & 1 & 0 \\
\end{pmatrix}
 $$

$$G = 
\begin{pmatrix}
    g_1 \\
    g_2 \\
    g_3 \\
\end{pmatrix}
\quad
G` = 
\begin{pmatrix}
    g_1 \\
    g_2 \\
    g_2 \oplus g_3 \\
\end{pmatrix}
$$



chyba kolejne zadanie 

\begin{align*}
    u_0 = 0g_1 \oplus 0g_2 \oplus 0g_3 = 00000
    u_1 = 1g_1 \oplus 0g_2 \oplus 0g_3 = 10110
    u_2 = 0g_1 \oplus 1g_2 \oplus 0g_3 = 01001
    u_3 = 1g_1 \oplus 1g_2 \oplus 0g_3 = 11110
    u_4 = 0g_1 \oplus 0g_2 \oplus 1g_3 = 11011
    u_5 = 1g_1 \oplus 0g_2 \oplus 1g_3 = 01101
    u_6 = 0g_1 \oplus 1g_2 \oplus 1g_3 = 10010
    u_7 = 1g_1 \oplus 1g_2 \oplus 1g_3 = 00100
\end{align*}

\subsection*{Problem 7?}

$$
H= 

\begin{pmatrix}
    0 & 1 & 0 & 0 & 1 \\
    1 & 0 & 0 & 1 & 0 \\
\end{pmatrix}
\quad
H = 
\begin{pmatrix}
   h_1 \\ 
   h_2 \\
\end{pmatrix}
$$

\begin{align*}
   v_0 = 0h_1 \oplus 0h_2 = 00000 
   v_1 = 1h_1 \oplus 0h_2 = 01001 
   v_2 = 0h_1 \oplus 1h_2 = 10010 
   v_3 = 1h_1 \oplus 1h_2 = 11011 
\end{align*}

$$S \in a_1g_1 \oplus a_2g_2 \oplus a_3g_3 = u $$
$$S^{\perp \in b_1h_1 \oplus b_2h_2 =v $$

$$v\cdot u = (b_1h_1 \oplus b_2h_2 )$$




\end{document}
