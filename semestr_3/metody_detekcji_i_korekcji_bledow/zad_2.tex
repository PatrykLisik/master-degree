
    \documentclass[12pt]{article}
    \usepackage[utf8]{inputenc}
    \usepackage{polski}
    \usepackage{enumitem}
    \usepackage{amsmath}

    \newcommand{\floor}[1]{\left\lfloor #1 \right\rfloor}

    \title{Zadanie 2}
    \author{Patryk Lisik}
    \date{\(8\) Lutego  2024}

    \begin{document}

    \maketitle
    \renewcommand{\abstractname}{Treść}

    \begin{abstract}
        Binarny kod blokowy składa się ze słów kodowych zapisanych w postaci systematycznej w tab. \ref{tab:kody}.
        \begin{table}[h]
            \centering
            \begin{tabular}{l}
                \hline
                c       \\  \hline
                000000  \\ 
                011100  \\
                101010  \\
                110110  \\
                110001  \\
                101101  \\
                011011  \\
                000111  \\ \hline
                
            \end{tabular}
            \caption{Tablica kodu blokowego} 
            \label{tab:kody}
        \end{table}
    \end{abstract}

    \begin{enumerate}[label=(\alph*)]
            \item Jakie jest tempo kodu R? 
            \item Zapisz macierz generującą $\mathbf{G}$ i kontroli parzystości $\mathbf{H}$ w postaci systematycznej.
            \item Jaka jest minimalna odległość Hamminga $d_{min}$ tego kodu?
            \item Jaka jest zdolność do poprawiania błędów $t$, a jaka zdolność do wykrywania błędów $l$ dla tego kodu?
            \item Oblicz syndrom $s$ dla odebranego sygnału $\mathbf{r} = (101011)$
    \end{enumerate}


    \section*{Rozwiązanie}
    
        \begin{table}[h]
            \centering
            \begin{tabular}{c|c}
                \hline
                Bity parzystości & Bity wiadomości       \\  \hline
                000 & 000  \\ 
                011 & 100  \\
                101 & 010  \\
                110 & 110  \\
                110 & 001  \\
                101 & 101  \\
                011 & 011  \\
                000 & 111  \\ \hline
            \end{tabular}
            \caption{Tablica kodu blokowego} 
        \end{table}

    \subsection*{a) Jakie jest tempo kodu R}
    \begin{equation*}
        R = \frac{k}{n} = \frac{3}{6} = \frac{1}{2}
    \end{equation*}

    \subsection*{b) Zapisz macierz generującą $\mathbf{G}$ i kontroli parzystości $\mathbf{H}$ w postaci systematycznej}
    \begin{align*}
        c = mG \\
        \mathbf{G} = (P \quad I_k) = \begin{pmatrix}
            1 & 1 & 0 \quad & 1 & 0 & 0 \\ 
            0 & 0 & 1 \quad & 0 & 1 & 0 \\ 
            0 & 1 & 1 \quad & 0 & 0 & 1 \\ 
        \end{pmatrix} \\ 
        \mathbf{H} = (I_{n-k} \quad P^T) = \begin{pmatrix}
            1 & 0 & 0 \quad & 1 & 1 & 0 \\ 
            0 & 1 & 0 \quad & 1 & 0 & 1 \\ 
            0 & 0 & 1 \quad & 0 & 1 & 1 \\ 
        \end{pmatrix}
    \end{align*}

    \subsection*{c) Jaka jest minimalna odległość Hamminga $d_{min}$ tego kodu?}

    \begin{table}[h]
        \centering
        \begin{tabular}{l|cccccccc}
            & 000000 & 011100 & 101010 & 110110 & 110001 & 101101 & 011011 & 000111 \\ \hline
            000000 & -      &        &        &        &        &        &        &        \\
            011100 & 3      & -      &        &        &        &        &        &        \\
            101010 & 3      & 4      & -      &        &        &        &        &        \\
            110110 & 4      & 3      & 3      & -      &        &        &        &        \\
            110001 & 3      & 4      & 4      & 3      & -      &        &        &        \\
            101101 & 4      & 3      & 3      & 4      & 3      & -      &        &        \\
            011011 & 4      & 3      & 3      & 4      & 3      & 4      & -      &        \\
            000111 & 3      & 4      & 4      & 3      & 4      & 3      & 3      & -     
        \end{tabular}
        \caption{Dystanse Hamminga pomiędzy słowami kodu}
        \label{tab:hammind-dist}
    \end{table}

    Jak pokazano w tabeli \ref{tab:hammind-dist} minimalny dystans Hamminga to 3. 

    \subsection*{d) Jaka jest zdolność do poprawiania błędów $t$, a jaka zdolność do wykrywania błędów $l$ dla tego kodu?}

    Zdolność do poprawiania błędów
    \begin{equation*}
        t= \floor{\frac{d_{min}-1}{2}} = 1
    \end{equation*}

    Zdolność do wykrywania błędów
    \begin{equation*}
        l = d_{min}-1 = 2
    \end{equation*}

    \subsection*{Oblicz syndrom $s$ dla odebranego sygnału $\mathbf{r} = (101011)$ i znajdź pozycję błędu} 
    $$ s = Hr^T $$
    \begin{align*}
        s=  \begin{pmatrix}
            1 & 0 & 0  & 1 & 1 & 0 \\ 
            0 & 1 & 0  & 1 & 0 & 1 \\ 
            0 & 0 & 1  & 0 & 1 & 1 \\ 
        \end{pmatrix} 
        \cdot
        \begin{pmatrix}
           1 \\ 
           0 \\ 
           1 \\ 
           0 \\ 
           1 \\ 
           1 \\ 
        \end{pmatrix} = 
        \begin{pmatrix}
            0 \\ 
            1 \\ 
            1 \\ 
        \end{pmatrix}
    \end{align*}

    Macierz syndromu jest 6 kolumną macierzy parzystości co oznacza że błąd jest na 6 pozycji odebranej wiadomości.








    \end{document}
