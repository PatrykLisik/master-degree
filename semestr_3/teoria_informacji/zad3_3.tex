\documentclass[12pt]{article}
\usepackage[utf8]{inputenc}
\usepackage{polski}
\usepackage{enumitem}
\usepackage{amsmath}

\title{Zadanie 3--3}
\author{Patryk Lisik}
\date{\(11\) Luty  2024}

\newcommand{\floor}[1]{\left\lfloor #1 \right\rfloor}

\begin{document}

\maketitle
\renewcommand{\abstractname}{Treść}

\begin{abstract}
    Zbuduj macierz Hadamarda rzędu 8 i na jej podstawie wyznacz kod Hadamarda rzędu 8. Jakie jest jego tempo R? 
    Jaka jest jego zdolność poprawiania i wykrywania błędów?
\end{abstract}


\section*{Rozwiązanie}
\subsection*{Macierz Hadamarda rzędu 8}
$H'$ jest macierzą rzędu 2n.
\begin{align*}
    H' = \begin{pmatrix}
        H & H \\
        H & -H \\
    \end{pmatrix}
\end{align*}

$$H_1 = (1) $$
$$H_2 = \begin{pmatrix}
    1 & 1 \\   
    1 & -1
\end{pmatrix}
$$

$$ 
H_4 = \begin{pmatrix}
    1 & 1  &  1  &  1 \\ 
    1 & -1 &  1  & -1 \\
    1 & 1  & -1  & -1 \\
    1 & -1 & -1  &  1 
\end{pmatrix}
$$

$$
H_8 = \begin{pmatrix}
    1 &  1 &  1  &  1 & 1  & 1  &  1   &  1 \\ 
    1 & -1 &  1  & -1 & 1  & -1 &  1   & -1 \\
    1 &  1 & -1  & -1 & 1  & 1  & -1   & -1 \\
    1 & -1 & -1  &  1 & 1  & -1 & -1   &  1 \\
    1 & 1  &  1  &  1 & -1 & -1 & -1  & -1 \\
    1 & -1 &  1  & -1 & -1 &  1 & -1  &  1 \\
    1 & 1  & -1  & -1 & -1 & -1 &  1  &  1 \\
    1 & -1 & -1  &  1 & -1 &  1 &  1  & -1 \\ 
\end{pmatrix}
$$

\subsection*{Kody Hadamrda 8 rzędu }

16 kodów Hadamara zbudowanych na bazie macierzy $H_8$ \\
(1 1 1 1 1 1 1 1) \\
(-1 -1 -1 -1 -1 -1 -1 -1) \\
(1 -1 1 -1 1 -1 1 -1) \\
(-1 1 -1 1 -1 1 -1 1) \\
(1 1 -1 -1 1 1 -1 -1) \\
(-1-1 1 1 -1 -1 1 1)  \\
(1 -1 -1 1 1 -1 -1 1)  \\
(-1 1 1 -1 -1 1 1 -1) \\
(1 1 1 1 -1 -1 -1 -1) \\
(-1 -1 -1 -1 1 1 1 1) \\
(1 -1 1 -1 -1 1 -1 1) \\
(-1 1 -1 1 1 -1 1 -1) \\
(1 1 -1 -1 -1 -1 1 1) \\
(-1 -1 1 1 1 1 -1 -1) \\
(1 -1 -1 1 -1 1 1 -1) \\
(-1 1 1 -1 1 -1 -1 1) \\

\subsection*{Tempo kodu R; Zdolność poprawiania i wykrywania błędów} 

Tempo kodu
$$ R = \frac{\log_2(2n)}{n} = \frac{\log_216}{8}=\frac{1}{2} $$ 
Zdolnośc poprawnia błędów
$$t=\frac{n}{4}-1=1 $$

Zdolność do wykrywania błędów
$$\mathcal{R} = d_{\min}-1 = 3$$



\end{document}
