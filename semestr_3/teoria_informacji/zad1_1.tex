\documentclass[12pt]{article}
\usepackage[utf8]{inputenc}
\usepackage{polski}
\usepackage{enumitem}
\usepackage{amsmath}


\title{Zadanie 1--1}
\author{Patryk Lisik}
\date{\(15\) Listopad  2023}

\begin{document}

\maketitle
\renewcommand{\abstractname}{Treść}

\begin{abstract}
Dyskretne niepamiętające źródło informacji $\mathcal{S}$ podaje wyniki losowania koloru z talii kart,
tj. znaku ze zbioru $S = \{ \heartsuit, \diamondsuit, \clubsuit, \spadesuit \} $. 
Do przekazania informacji z tego źródła zastosowano kod binarny $\mathcal{C} = \{ 0, 11,110,1111 \}$ 
przypisujący kolorom kart następujące słowa kodowe:

    \begin{figure}[h]
\centering
\begin{tabular}{ccccc}
    $\mathcal{S}_i$ & $\heartsuit$   & $\diamondsuit$   & $\clubsuit$   & $\spadesuit$ & \hline
    $\omega_i$      & 0              & 10               & 110           & 111 &
\end{tabular}
\label{tab:codes}

\end{figure}

    Wyznacz dla tego kodu zbiory $\mathcal{C}_i$, $i=0,1....i$, i $\mathcal{C}_{\infty}$, wykorzystywane w twierdzeniu
    Sardanisa--Pattersona, i sprawdź, czy powyższy kod jest jednoznacznie dekodowalny i czy jest natychmiastowy(tzn.
     nie wprowadza opóźnień).
\end{abstract}


\section*{Rozwiązanie}

Wyznaczenie $\mathcal{C}_i$.
\begin{flalign*}
    \mathcal{C}_0 = \{0,10,110,111\} \\
    \mathcal{C}_1 = \mathcal{C}_2 = \cdots = \mathcal{C}_{\infty} = \emptyset
\end{flalign*}

Na podstawie twierdzenia Sardanisa--Pattersona kod $\mathcal{S}$ jest jednoznacznie dekodowalny,
ponieważ zbiory $\mathcal{C}_0$ i $\mathcal{C}_{\infty}$ są rozłączone.   

Kod jest natychmiastowy, ponieważ jest prefiksowy($\mathcal{C}_1 = \emptyset $).
\end{document}
