\documentclass[12pt]{article}
\usepackage[utf8]{inputenc}
\usepackage{polski}
\usepackage{enumitem}
\usepackage{amsmath}
\usepackage{epsdice}
\usepackage[table,xcdraw]{xcolor}

\title{Zadanie 1--2}
\author{Patryk Lisik}
\date{\(16\) Listopad  2023}

\begin{document}

\maketitle
\renewcommand{\abstractname}{Treść}

\begin{abstract}
    Zbuduj binarny kod Huffmana dla źródła $\mathcal{S}$ podającego wynik rzutu nieuczciwą kostką 
    $S = \{ \epsdice{1}, \epsdice{2}, \epsdice{3}, \epsdice{4}, \epsdice{5}, \epsdice{6} \}$ 
    z prawdopodobieństwami:
    \begin{figure}[h]
\centering
\begin{tabular}{ccccccc}
    $s_i$ & $\epsdice{1}$ & $\epsdice{2}$ & $\epsdice{3}$   & $\epsdice{4}$ & $\epsdice{5}$ & $\epsdice{1}$  &\hline
    $\omega_i$      & 0.3  & 0.25 & 0.2 & 0.1 & 0.08 & 0.07 &
\end{tabular}
\label{tab:codes}

\end{figure}

    i znajdź średnią długość słowa. Na ile  długość słów i średnia długość słowa są wyznaczone jednoznacznie?
\end{abstract}


\section*{Rozwiązanie}
Budowanie binarnego kodu Huffama

\begin{table}[h]
\begin{tabular}{|l|l|l|l|l|l|l|l|l|l|l|l|}
\hline
 &
  \cellcolor[HTML]{FFCCC9}$S^{(5)}$ &
   \cellcolor[HTML]{FFCCC9}$S^{(4)}$ &
  \cellcolor[HTML]{FFCCC9}$S'''$ &
  \cellcolor[HTML]{FFCCC9}$S_1$ &
  \cellcolor[HTML]{FFCCC9}$S_2$ &
  \cellcolor[HTML]{FFCCC9}$S''$ &
  \cellcolor[HTML]{FFCCC9}$S_3$ &
  \cellcolor[HTML]{FFCCC9}$S' $&
  \cellcolor[HTML]{FFCCC9}$S_4$&
  \cellcolor[HTML]{FFCCC9}$S_5$ &
  \cellcolor[HTML]{FFCCC9}$S_6$ \\ \hline
\cellcolor[HTML]{FFCCC9}S   &   &      &      & 0.3 & 0.25 &      & 0.2 &      & 0.1 & 0.08 & 0.07 \\ \hline
\cellcolor[HTML]{FFCCC9}S'  &   &      &      & 0.3 & 0.25 &      & 0.2 & 0.15 & 0.1 &      &      \\ \hline
\cellcolor[HTML]{FFCCC9}S'' &   &      &      & 0.3 & 0.25 & 0.25 & 0.2 &      &     &      &      \\ \hline
\cellcolor[HTML]{FFCCC9}S''' &   &      & 0.45 & 0.3 & 0.25 &      &     &      &     &      &      \\ \hline
\cellcolor[HTML]{FFCCC9}S^{(4)}  &   & 0.55 & 0.45 &     &      &      &     &      &     &      &      \\ \hline
\cellcolor[HTML]{FFCCC9}S^{(5)}  & 1 &      &      &     &      &      &     &      &     &      &      \\ \hline
\end{tabular}
\end{table}

\begin{table}[t]
\begin{tabular}{|l|l|l|l|l|l|l|l|l|l|l|l|}
\hline
 &
  \cellcolor[HTML]{FFCCC9}S5 &
  \cellcolor[HTML]{FFCCC9}S4 &
  \cellcolor[HTML]{FFCCC9}S''' &
  \cellcolor[HTML]{FFCCC9}S1 &
  \cellcolor[HTML]{FFCCC9}S2 &
  \cellcolor[HTML]{FFCCC9}S'' &
  \cellcolor[HTML]{FFCCC9}S3 &
  \cellcolor[HTML]{FFCCC9}S' &
  \cellcolor[HTML]{FFCCC9}S4 &
  \cellcolor[HTML]{FFCCC9}S5 &
  \cellcolor[HTML]{FFCCC9}S6 \\ \hline
\cellcolor[HTML]{FFCCC9}S   &   &   &   & 00 & 01 &    & 11 &     & 101 & 1000 & 1001 \\ \hline
\cellcolor[HTML]{FFCCC9}S1  &   &   &   & 00 & 01 &    & 11 & 100 & 101 &      &      \\ \hline
\cellcolor[HTML]{FFCCC9}S'' &   &   &   & 00 & 01 & 10 & 11 &     &     &      &      \\ \hline
\cellcolor[HTML]{FFCCC9}S'' &   &   & 1 & 00 & 01 &    &    &     &     &      &      \\ \hline
\cellcolor[HTML]{FFCCC9}S4  &   & 0 & 1 &    &    &    &    &     &     &      &      \\ \hline
\cellcolor[HTML]{FFCCC9}S5  & 1 &   &   &    &    &    &    &     &     &      &      \\ \hline
\end{tabular}
\end{table}

Obliczanie średniej długości słowa
\begin{table}[t]
\begin{tabular}{|l|l|l|l|l|}
\hline
\rowcolor[HTML]{FFCCC9} 
    Kodowany symbol       & Kod  & $p_i$  & Długość słowa & $L(c)$  \\ \hline
   \epsdice{1}        & 00   & 0.3  & 2 & 0.6       \\ \hline
   \epsdice{2}        & 01   & 0.25 & 2 & 0.5      \\ \hline
   \epsdice{3}       & 11   & 0.2  & 2 & 0.4      \\ \hline
   \epsdice{4}      & 101  & 0.1  & 3 & 0.3      \\ \hline
   \epsdice{5}      & 1000 & 0.08 & 4 & 0.32      \\ \hline
   \epsdice{6}    & 1001 & 0.07 & 3 & 0.21      \\ \hline
                      &      &      &         &     \\ \hline
    Średnia długość słowa &      &      &   & $\sum$ 2.33      \\ \hline
\end{tabular}
\label{tab:my-table}

Długości słów są wyznaczone jednoznacznie oraz co za tym idzie średnia długość słów jest wyznaczona jednoznacznie.
\end{table}

\subsection*{Entropia źródła}

\begin{multline*}
    H(\mathcal{S}) = \sum p_1 \log_2 \frac{1}{p_i} =\\
    -(0.3 \log_2 0.3 + 0.25 \log_2 0.25 + 0.1 \log_2 0.1 + 0.08 \log_2 0.08 + 0.07 \log_2 0.07 )\\
    \approx 2.37 
\end{multline*}

\subsection*{Efektywność kodu Huffmana}
 
Efektywność kodu Huffmana:
$$\eta = \frac{H(\mathcal{S})}{L(\mathcal{C})}* 100\% = \frac{2.33}{2.37} \approx 98.31 \% $$ 

\end{document}
