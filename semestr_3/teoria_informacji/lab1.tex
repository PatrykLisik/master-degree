\documentclass[12pt]{article}
\usepackage[utf8]{inputenc}
\usepackage{polski}
\usepackage{enumitem}
\usepackage{amsmath}


\title{Teoria informacji -- Lab1}
\author{Patryk Lisik}
\date{\(19\) Listopad  2023}

\begin{document}

\maketitle


\section*{Problem 1}

Dane jest niepamiętające dyskretne źródło informacji z alfabetem $A = \{a,b,c,d\}$ i prawdopodobieństwem 
nadania każdego znaku odpowiednio $P(X=a)=\frac{1}{2}, P(X=b)=P(X=c)=\frac{1}{8}, P(X=d) = \frac{1}{2}$. 
Jaka jest entropia źródła?

\begin{table}[h]
\begin{tabular}{|l|l|l|l|l|}
\hline
x\_i & a   & b   & c   & d   \\ \hline
p\_i & 1/2 & 1/8 & 1/8 & 1/4 \\ \hline
I\_i & 1b  & 3b  & 3b  & 2b  \\ \hline
\end{tabular}

\end{table}
Ilość informacji I
$I_i = log_2\frac{1}{p_i} $

$$H(X) = \sum p_iI_i = 1.75b $$

\section*{Problem 2}
Źródło o szerokości pasma $W = 4000Hz$ zostało poddane próbkowaniu z częstotliwością Nyquista.
Przyjmując 


\begin{table}[h]
\begin{tabular}{|l|l|l|l|l|l|}
\hline
    x\_i & -2   & -1   & 0   & 1 & 2   \\ \hline
    p\_i & 1/2 & 1/4 & 1/8 & 1/16 & 1/16 \\ \hline
    I\_i & 1b  & 3b  & 3b  & 2b & 2b \\ \hline
\end{tabular}

\end{table}

H(X) = 1.875b

Musimy próbkowań z częstotliwością dwukrotnie większą niż źródło = 8000 hz

8000 * 1.875b = 15000 b/s

\section*{Problem 3}


\begin{table}[h]
\begin{tabular}{|l|l|l|l|l|l|}
\hline
x\_i & A   & B   & C   & C   \\ \hline
p\_i & 1/3 & 1/3 & 1/6 & 1/6 \\ \hline
I\_i & $\log_23$  & $\log_23$  & $1+\log_23$  & $1+\log_23$  \\ \hline
\end{tabular}

\end{table}

$$ H(X) = \frac{1}{3} + \log_23 \approx = 1.918b $$
$$log_23 \approx 1.585 $$

\section*{Problem 3}

$$H(X^2) = \frac{1}{4}\cdot 2b+\frac{2}{8}\cdot 3b + \frac{5}{16}\cdot 4b + \frac{4}{32} \cdot 5b + 
\frac{4}{64} 6b =3.5b = H(X)  $$

\section*{Problem 4}
$$P= 
\begin{pmatrix}
    \frac{3}{4} & \frac{1}{4} \\
    \frac{1}{8} & \frac{7}{8} 
\end{pmatrix}
$$
$$p=\left( \frac{4}{5} \frac{1}{5}  \right) $$
$$H(X) = \frac{4}{5}\log_2(\frac{5}{4}) + \frac{1}{5} \log_25 = \log_25 - \frac{8}{5}  $$

Prawdopodobieńśtwa wyjścia wyjścia:

$$q = pP = (\frac{4}{5} \frac{1}{5} )   
\begin{pmatrix}
    \frac{3}{4} & \frac{1}{4} \\
    \frac{1}{8} & \frac{7}{8} 
\end{pmatrix}
= \left( \frac{25}{40} \frac{15}{40}  \right) 
= \left( \frac{5}{8} \frac{3}{8}  \right) 
$$

Entropia wyjściowa:

$$H(Y) = \frac{5}{8}\log_2\frac{8}{5} + \frac{3}{8}\log_2\frac{8}{3} = 0.954b $$

Macierz prawdopodobieńs włącznych 

$$ R =
\begin{pmatrix}
    \frac{4}{5} & 0 \\
    0 & \frac{1}{5} 
\end{pmatrix}
\begin{pmatrix}
    \frac{3}{4} & \frac{1}{4} \\
    \frac{1}{8} & \frac{7}{8} 
\end{pmatrix}=
\begin{pmatrix}
    \frac{3}{5} & \frac{1}{5} \\
    \frac{1}{40} & \frac{7}{40} 
\end{pmatrix}
$$
Entropia łączna
Wynik powinien sumować się do 1. 

$$H(X,Y)= \frac{3}{5}\log_2\frac{5}{3} + \frac{1}{5}\log_25 + \frac{1}{40}\log_2 40 + \frac{7}{40}\frac{40}{7} =
= \frac{3}{5} + \log25 - \frac{3}{5}\log_23 - \frac{7}{40}\log_27 \approx 1.780 b
$$

Entropia szumu :

$$H(Y|X) = \frac{3}{5}\log_2 \frac{4}{3} + \frac{1}{5}\log_24 + \frac{1}{40}\log_2{8} + \frac{7}{40}\log_2 \frac{7}{8}=
= \frac{6}{5} - \frac{3}{5}\log_23 + \frac{2}{5} + \frac{3}{40} + \frac{21}{40} - \frac{7}{40}\log_27 = 
\frac{11}{5} - \frac{3}{5}\log_23 - \frac{7}{40}\log_2 7
$$

Macierz Q / Macierz wstecz

$$Q = 
\begin{pmatrix}
    \frac{3}{5} & \frac{1}{5} \\
    \frac{1}{40} & \frac{7}{40} 
\end{pmatrix}
\begin{pmatrix}
    \frac{8}{5} & 0 \\
    0 & \frac{8}{3} 
\end{pmatrix}=
\begin{pmatrix}
    \frac{24}{25} & \frac{8}{15} \\
    \frac{1}{25} & \frac{7}{15} 
\end{pmatrix}
$$

Ważne 

Tu nie ma mnożenie macierzowego 
$$p_iP_{ij} =R_{ij} = Q_{ij}q_j  $$
$$q_j = \sum_i p_iP_{ij} $$


Ekwiwokacja
$$H(X|Y) = \frac{3}{5}\log \frac{25}{24} + \frac{1}{5}\log{15}{8} + \frac{1}{40}\log_2 25 + \frac{7}{40}\log_2\frac{15}{7}
= \frac{6}{5}\log_2 5 - \frac{3}{5}\log_23 - \frac{9}{5}+\frac{1}{5}\log_23+\frac{1}{20}\log_2{5}+\frac{7}{40}\log_25+\frac{7}{40}\log_23-\frac{7}{40}\log_27
=\frac{65}{40}\log_25 - \frac{12}{5} - \frac{9}{40}\log_23 - \frac{7}{40}\log_27 \approx 0.525b
$$

Informacja wzajemna
$$ I(X,Y) = H(X)+ H(Y) - H(X,Y) = \frac{4}{5} + \frac{9}{40}\log_23 - \frac{5}{8}\log_25 + \frac{7}{40}\log_27 \approx 0.197b $$

\section*{Problem 7}
Wyznacz pojemnośc kanału 

$$
P=
\begin{pmatrix}
    \frac{1}{2} & \frac{1}{4} & \frac{1}{4} \\
    \frac{1}{4} & \frac{1}{2} & \frac{1}{4} \\
    \frac{1}{4} & \frac{1}{4} & \frac{1}{2} \\
\end{pmatrix}

$$

$$p = (x y z ) $$
$$ x+y+z = 1 $$
$$ C = max I(X,Y) = max [H(X) + H(Y) - H(X,Y) ] $$ 
$$q=(x y z ) 
\begin{pmatrix}
    \frac{1}{2} & \frac{1}{4} & \frac{1}{4} \\
    \frac{1}{4} & \frac{1}{2} & \frac{1}{4} \\
    \frac{1}{4} & \frac{1}{4} & \frac{1}{2} \\
\end{pmatrix}
= \left( \frac{2x+y+z}{4} \frac{x+2y+z}{4} \frac{x + y +2z}{4}  \right)
$$ 


$$ H(X) = -x \log_2 X - y\log y - z \log z $$ 
$$ H(Y) =  -\frac{2x+y+z}{4}\log_2 \frac{2x+y+z}{4} - \frac{x+2y+z}{4}\log_2 \frac{x+2y+z}{4} - \frac{x + y +2z}{4}\log_2 \frac{x + y +2z}{4}  $$


\end{document}
