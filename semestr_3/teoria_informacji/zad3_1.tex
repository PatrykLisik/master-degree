\documentclass[12pt]{article}
\usepackage[utf8]{inputenc}
\usepackage{polski}
\usepackage{enumitem}
\usepackage{amsmath}
\usepackage{tikz}

\title{Zadanie 3--1}
\author{Patryk Lisik}
\date{\(11\) Luty  2024}

\begin{document}

\maketitle
\renewcommand{\abstractname}{Treść}

\begin{abstract}
    Wylicz wszystkie słowa kodowe kodu Hamminga $\mathcal{H}_7$ i sprawdź, że jedgo odległość minimalna to $ d=3 $.
\end{abstract}


\section*{Rozwiązanie}
W kodach liniowych każde słowo kodowe jest sumą $\mod 2$ dwóch innych słów kodowych. 
Oznacza to że możemy dlatego minimalną odległość $d$ można wyrazić jako odległość od wektora zerowego lub 
jako ilość $1$ w danym słowie kodowym.

\begin{table}[h]
    \centering
    \begin{tabular}{ccc}
        wiadomość & kod       & $d$ \\ \hline
        0000      & 0000000   &  - \\ 
        0001      & 1101001   &  4 \\ 
        0010      & 0101010   &  3 \\ 
        0011      & 1000011   &  3 \\ 
        0100      & 1000110   &  3 \\ 
        0101      & 0100101   &  3 \\ 
        0110      & 1100110   &  4 \\ 
        0111      & 0001111   &  4 \\ 
        1000      & 1110000   &  4 \\ 
        1001      & 0011001   &  4 \\ 
        1010      & 1011010   &  4 \\ 
        1011      & 0110011   &  4 \\ 
        1100      & 0111100   &  4 \\ 
        1101      & 1010101   &  3 \\ 
        1110      & 0010110   &  3 \\ 
        1111      & 1111111   &  7 \\ 
    \end{tabular}
    \caption{Wiadomości kody i ich odległość od wektora zerowego)}
    \label{tab}
\end{table}

Jak pokazano w tabeli \ref{tab} $d_{\min}=3$.

\end{document}
