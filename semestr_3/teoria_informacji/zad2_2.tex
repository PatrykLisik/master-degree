\documentclass[12pt]{article}
\usepackage[utf8]{inputenc}
\usepackage{polski}
\usepackage{enumitem}
\usepackage{amsmath}
\usepackage{tikz}

\title{Zadanie 2--2}
\author{Patryk Lisik}
\date{\(12\) Grudnia  2023}

\begin{document}

\maketitle
\renewcommand{\abstractname}{Treść}

\begin{abstract}
Pokaż, że dla niesymetrycznego kanały binarnego z rysunku 1, gdzie 
    $$ Pr(X=0)=x \quad  Pr(X=1)=1-x $$
    oraz 
    \begin{flalign*}
        & \text{Pr}(Y=0|X=0)=1-p & \text{Pr}(Y=1|X=0)=p   &\\ 
        & \text{Pr}(Y=0|X=1)=q   & \text{Pr}(Y=1|X=1)=1-q &\\
    \end{flalign*}
    informacja wzajemna dana jest wzorem
    $$I(X;Y) = \Omega((1-p)x+q(1-x))-x\Omega (p)-(1-x)\Omega(q) $$
    gdzie 
    $$\Omega(x) = -x\log_2x-(1-x)\log_2(1-x) $$

\end{abstract}


\section*{Rozwiązanie}


\begin{flalign*}
 & I(X;Y) = H(X) - H(X|Y) \\
    & H(X) =  -(x\log_2x +(1-x)\log_2(1-x)) & H(X|Y) = q_0H(X|y_0) + q_1H(X|y_1) \\ 
\end{flalign*}

\begin{flalign*}
    & P_{ij} = 
\begin{pmatrix}
    1-p & p \\ 
    q   & 1-q \\
\end{pmatrix} \\
    & q_0 = x(1-p)+(1-x)q  = -px - qx + q +x = q - x(p+q+1) \\ 
    & q_1 = x(1-q) + xp  = x-qx+xp  = x(1-q+p)\\
    & Q_{ij} = 
    \begin{pmatrix}
        \frac{p_0}{q_0}P_{00} & \frac{p_1}{q_0}P_{10} \\    
        \frac{p_0}{q_1}P_{01} & \frac{p_1}{q_1}P_{11} \\
    \end{pmatrix}= 
    \begin{pmatrix}
        \frac{x(1-p)}{x(1-p)(1-x)q} & \frac{(1-x)(p)}{x(1-p)(1-x)q} \\[4pt]  
        \frac{xq}{x(1-q) + xp} & \frac{(1-x)(1-q)}{x(1-q) + xp} \\
    \end{pmatrix}
\end{flalign*}

\begin{flalign*}
    & H(X|y_j) = - \sum_i Q_{ij}\log_2 Q_{ij} \\ 
    & H(X|y_0) = -Q_{00}\log_2Q_{00} - Q_{10}\log_2 Q_{10}  \\
    & H(X|y_1) = -Q_{01}\log_2 Q_{01} - Q_{11} \log_2 Q_{11} \\
\end{flalign*}

\begin{flalign*}
    & H(X|Y) = q_0H(X|y_0) + q_1H(X|y_1)  \\ 
    & = q_0 \left(
    \frac{p_0}{q_0}P_{00} \log_2  \frac{q_0}{p_0P_{00}} + 
    \frac{p_1}{q_0}P_{10} \log_2  \frac{q_0}{p_1P_{10}}
    \right)
     +  q_1 \left(
    \frac{p_0}{q_1}P_{01} \log_2  \frac{q_1}{p_0P_{01}} + 
    \frac{p_1}{q_1}P_{11} \log_2  \frac{q_1}{p_1P_{11}}
    \right) \\ 
    & = p_0P_{00}\log_2 q_0 - p_0P_{00} \log_2 p_0P_{00} \\
    & + p_1P_{10}\log_2 q_0 - p_1P_{10}\log_2 p_1P_{10} \\ 
    & + p_0P_{01} \log_2 q_1 - p_0P_{01} \log_2 p_0P_{01} \\
    & + p_1P_{11}\log_2 q_1 - p_1P_{11}\log_2p_1P_{11}
\end{flalign*}


\begin{flalign*}
    &p_0 =x \quad p_1 = 1-x \\
    & I(X;Y) = -x\log_2x - (1-x)\log_2(1-x) \\ 
    & - x(1-p)\log_2 (x(1-p) + (1-x)q)  + x(1-p) \log_2 x(1-p) \\
    & - (1-x)q\log_2 (x(1-p) + (1-x)q) + (1-x)q\log_2 (1-x)q \\ 
    & - x q \log_2 (x(1-q)+xp) + xq \log_2 xq \\ 
    & + (1-x)(1-q)\log_2 (x(1-q)+xp) + (1-x)(1-q)\log_2 ((1-x)(1-q)) \\
    & \\
    &\Omega((1-p)x+q(1-x)) = \\
    & -(1-p)x+q(1-x) \log_2 ((1-p)x+q(1-x)) -(1-(1-p)x+q(1-x))  
\end{flalign*}



\end{document}
