\documentclass[12pt]{article}
\usepackage[utf8]{inputenc}
\usepackage{polski}
\usepackage{enumitem}
\usepackage{amsmath}
\usepackage{tikz}

\title{Zadanie 2--2}
\author{Patryk Lisik}
\date{\(12\) Grudnia  2023}

\begin{document}

\maketitle
\renewcommand{\abstractname}{Treść}

\begin{abstract}
Pokaż, że dla niesymetrycznego kanały binarnego z rysunku 1, gdzie 
    $$ Pr(X=0)=x \quad  Pr(X=1)=1-x $$
    oraz 
    \begin{flalign*}
        & \text{Pr}(Y=0|X=0)=1-p & \text{Pr}(Y=1|X=0)=p   &\\ 
        & \text{Pr}(Y=0|X=1)=q   & \text{Pr}(Y=1|X=1)=1-q &\\
    \end{flalign*}
    informacja wzajemna dana jest wzorem
    $$I(X;Y) = \Omega((1-p)x+q(1-x))-x\Omega (p)-(1-x)\Omega(q) $$
    gdzie 
    $$\Omega(x) = -x\log_2x-(1-x)\log_2(1-x) $$

\end{abstract}


\section*{Rozwiązanie}
$$I(X;Y) = H(X) - H(X|Y) $$
$$H(X) =  -(x\log_2x +(1-x)\log_2(1-x)$$

$$H(X|Y) = q_0H(X|y_0) + q_1H(X|q_1) $$

$$P_{ij} = 
\begin{pmatrix}
    1-p & p \\ 
    q   & 1-q \\
\end{pmatrix}
$$

\begin{flalign*}
    & q_0 = x(1-p)+(1-x)q 
    & q_1 = x(1-q) + xp
\end{flalign*}


\end{document}
