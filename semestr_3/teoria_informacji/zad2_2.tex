\documentclass[12pt]{article}
\usepackage[utf8]{inputenc}
\usepackage{polski}
\usepackage{enumitem}
\usepackage{amsmath}
\usepackage{tikz}

\title{Zadanie 2--2}
\author{Patryk Lisik}
\date{\(12\) Grudnia  2023}

\begin{document}

\maketitle
\renewcommand{\abstractname}{Treść}

\begin{abstract}
Pokaż, że dla niesymetrycznego kanały binarnego z rysunku 1, gdzie 
    $$ Pr(X=0)=x \quad  Pr(X=1)=1-x $$
    oraz 
    \begin{flalign*}
        & \text{Pr}(Y=0|X=0)=1-p & \text{Pr}(Y=1|X=0)=p   &\\ 
        & \text{Pr}(Y=0|X=1)=q   & \text{Pr}(Y=1|X=1)=1-q &\\
    \end{flalign*}
    informacja wzajemna dana jest wzorem
    $$I(X;Y) = \Omega((1-p)x+q(1-x))-x\Omega (p)-(1-x)\Omega(q) $$
    gdzie 
    $$\Omega(x) = -x\log_2x-(1-x)\log_2(1-x) $$

\end{abstract}


\section*{Rozwiązanie}

\begin{flalign*}
    & I(X;Y) = H(Y) - H(Y|X)
    & \\
    & q_0 = x(1-p)+(1-x)q \\
    & q_1 = (1-x)(1-q)+xp = 1- x+px-q-xq \\
    & H(Y) = -\sum_j q_j\log_2 q_j = -q_0\log_2 q_0 - q_1\log_2 q_1 \\
    & = x(1-p)+q(1-x)\log_2 (x(1-p)+q(1-x)) - (1-x+px-q-xq) \log_2 (1-x+px-q-xq) \\ 
    & = \Omega((1-p)x+q(1-x))
    & \\
\end{flalign*}

\begin{flalign*}
    & H(Y|X) = \sum_i p_1H(Y|x_1) = p_0H(Y|x_0) + p_1H(Y|x_1) \\
    & p_0H(Y|x_0) = x \left(\sum_i  \text{Pr}(y_1|x_0) \log_2 \text{Pr}(y_1|x_0) \right)\\
    & = x(\text{Pr}(y_0|x_0) \log_2 \text{Pr}(y_0|x_0) + \text{Pr}(y_1|x_0) \log_2 \text{Pr}(y_1|x_0)) \\
    & = x\Omega(p)
\end{flalign*}


\begin{flalign*}
    & p_1H(Y|x_1) = (1-x)\left( \sum_i \text{Pr}(y_i|x_1) \log_2 \text{Pr}(y_i|x_1) \right) \\
    & (1-x)(\text{Pr}(y_0|x_1) \log_2 \text{Pr}(y_0|x_1) + \text{Pr}(y_1|x_1) \log_2 \text{Pr}(y_1|x_1) ) \\
    & = (1-x)\Omega(q)
\end{flalign*}


\end{document}
