\documentclass[12pt]{article}
\usepackage[utf8]{inputenc}
\usepackage{polski}
\usepackage{enumitem}
\usepackage{amsmath}
\usepackage{tikz}

\title{Zadanie 2--1}
\author{Patryk Lisik}
\date{\(12\) Grudnia  2023}

\begin{document}

\maketitle
\renewcommand{\abstractname}{Treść}

\begin{abstract}
Dla dwóch jednakowych binarnych kanałów symetrycznych połączonych
szeregowo i przedstawionych na rysunku 1 wyznacz macierz prawdopodobieństw
    warunkowych w przód $P_{ij} = \text{Pr}(Y = y_j |X = x_i)$ kanału $(x_i = 0, 1;
    y_j = 0, 1)$. Jakie są prawdopodobieństwa wyjściowe $q_j = \text{Pr}(Y = y_j)$ kanału, jeżeli
    $p_0 = \text{Pr}(X = 0) = x, p_1 = \text{Pr}(X = 1) = 1 - x$? Wyznacz macierz prawdopodobieństw
    warunkowych wstecz $Q_{ij} = \text{Pr}(X = x_i|Y = y_j)$ kanału oraz macierz
    prawdopodobieństw łącznych $\text{R}_{ij} = \text{Pr}(X = x_i, Y = y_j)$.

%    \begin{figure}[h]
%    \begin{tikzpicture}
%        \draw[gray, thick] (0,1) -- (2,1);
%        \draw[gray, thick] (0,0) -- (2,0);
%    \end{tikzpicture}
%    \end{figure}
\end{abstract}


\section*{Rozwiązanie}
Macierz w przód $P_{ij}$
$$
\begin{pmatrix}
    (1-p)(1-p)+p\cdot p           & (p\cdot(1-p)) + (1-p)\cdot p \\
     (p\cdot(1-p)) + (1-p)\cdot p & (1-p)(1-p)+p\cdot p          \\
\end{pmatrix}=
\begin{pmatrix}
    2p^2 - 2p + 1 & 2p-2p^2        \\ 
    2p-2p^2       & 2p^2 - 2p + 1  \\ 
\end{pmatrix}
$$
Łatwo zauważyć że $P=-2p^2+2p$
\\ \\
Prawdopodobieństwo wyjściowe
\begin{flalign*}
    & q_0 = xP_{00}+ (1-x)P_{10} = 4p^2-2p^2-4px+2p+x \\ 
    & q_1 = (1-x)P_{11} + xP_{01} = -2p^2+2p+x-1 \\
\end{flalign*}

Macierz prawdopodobieństwo wstecz $Q_{ij}$

$$Q_{ij} = \frac{p_i}{q_j}P_{ij} $$

\begin{flalign*}
    Q = 
    \begin{pmatrix}
        \frac{p_0}{q_0}P_{00} & \frac{p_1}{q_0}P_{10} \\    
        \frac{p_0}{q_1}P_{01} & \frac{p_1}{q_1}P_{11} \\
    \end{pmatrix} = 
    \begin{pmatrix}
        \frac{x(2p^2-2p+1)}{4p^2x-2p^2-4px+2p+x} & \frac{(1-x)(2p-2p^2)}{4p^2x-2p^2-4px+2p+x} \\[5pt]
        \frac{x(2p-2p^2)}{-2p^2+2p+x-1}          & \frac{(1-x)(2p^2-2p+1)}{-2p^2+2p+x-1}
    \end{pmatrix}
\end{flalign*}

Macierz prawdopodobieństwo łącznych:
$$R_{ij} = p_iP_{ij} $$

$$
R = 
\begin{pmatrix}
    p_{0}P_{00} & p_{1}P_{10} \\ 
    p_0P_{01}   & p_1P_{11}
\end{pmatrix} = 
\begin{pmatrix}
    x(2p^2-2p+1) &  (1-x)(2p-2p^2) \\
    x(2p-2p^2)   &  (1-x)(2p^2-2p+1)
\end{pmatrix}
$$
\end{document}
