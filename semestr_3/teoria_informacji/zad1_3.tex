\documentclass[12pt]{article}
\usepackage[utf8]{inputenc}
\usepackage{polski}
\usepackage{enumitem}
\usepackage{amsmath}
\usepackage{epsdice}
\usepackage[table,xcdraw]{xcolor}

\title{Zadanie 1--3}
\author{Patryk Lisik}
\date{\(19\) Listopad  2023}

\begin{document}

\maketitle
\renewcommand{\abstractname}{Treść}

\begin{abstract}
    Dyskretne niepamiętające źródło informacji $\mathcal{S}$ podaje wynik rzutu nieuczciwą kostką 
    $S = \{ \epsdice{1}, \epsdice{2}, \epsdice{3}, \epsdice{4}, \epsdice{5}, \epsdice{6} \}$ 
    z prawdopodobieństwami:
    \begin{figure}[h]
\centering
\begin{tabular}{ccccccc}
    $s_i$ & $\epsdice{1}$ & $\epsdice{2}$ & $\epsdice{3}$   & $\epsdice{4}$ & $\epsdice{5}$ & $\epsdice{1}$  &\hline
    $\omega_i$      & 0.3  & 0.25 & 0.2 & 0.1 & 0.08 & 0.07 &
\end{tabular}
\label{tab:codes}

\end{figure}

    i znajdź średnią długość słowa $L$ i efektywność $\eta$ dla kodu Shannona--Fano $\mathcal{D}$ i porównaj wyniki z wartościami dla kodu optymalnego.
\end{abstract}


\section*{Rozwiązanie}
Materiały wykładowe nie zawierają algorytmu budowania kodów Shannona--Fano. Wykorzystano następujący algorytm budowania kodów Shannona-Fano zdefiniowany w "Wprowadzenie do kompresji danych" -- Adam Drozdek.
Definicja algorytmu znajduje się w rozdziale 2.2 -- Kodowanie Shannona--Fano.

Algorytm kodowania jest następujący:\\
$s$ – ciąg symboli ze zbioru posortowanych wg prawdopodobieństw\\ 
Funkcja $Shannon-Fano(s)$: \\
Jeśli $s$ zawiera dwa symbole, do słowa kodu pierwszej litery dodaj 0, do słowa kodu drugiej litery – 1.\\
W przeciwnym razie (s zawiera więcej niż dwa symbole) podziel s na dwa podciągi $s1$ i $s2$ tak, 
żeby różnica między sumą prawdopodobieństw liter z $s1$ i $s2$ była najmniejsza.
Do słów kodu symboli z $s1$ dodaj 0, do kodów symboli z $s2$ – 1.
Wywołaj rekurencyjne funkcje: $Shannon-Fano(s1)$ oraz $Shannon-Fano(s2)$.



\subsection*{Średnia długość słowa}
Obliczanie średniej długości słowa w kodowaniu Shannona--Fano. 
\begin{table}[h]
\begin{tabular}{|l|l|l|l|l|}
\hline
\rowcolor[HTML]{FFCCC9} 
    Kodowany symbol       & Kod  & $p_i$  & Długość słowa & $L(c)$  \\ \hline
   \epsdice{1}        & 00   & 0.3  & 2 & 0.6       \\ \hline
   \epsdice{2}        & 01   & 0.25 & 2 & 0.5      \\ \hline
   \epsdice{3}       & 101   & 0.2  & 3 & 0.6     \\ \hline
   \epsdice{4}      & 1100  & 0.1  & 4 & 0.4      \\ \hline
   \epsdice{5}      & 1110 & 0.08 & 4 & 0.32      \\ \hline
   \epsdice{6}    & 1111 & 0.07 & 4 & 0.21      \\ \hline
                      &      &      &         &     \\ \hline
    Średnia długość słowa &      &      &   & $\sum$ 2.63      \\ \hline
\end{tabular}
\label{tab:my-table}
\end{table}
\subsection*{Entropia źródła}

\begin{multline*}
    H(\mathcal{S}) = \sum p_1 \log_1 \frac{1}{p_i} =\\
    -(0.3 \log_2 0.3 + 0.25 \log_2 0.25 + 0.1 \log_2 0.1 + 0.08 \log_2 0.08 + 0.07 \log_2 0.07 )\\
    \approx 2.37
\end{multline*}

\subsection*{Porównanie wydajności kodów}
 
Efektywność kodu Huffmana:
$$\eta = \frac{H(\mathcal{S})}{L(\mathcal{C})}* 100\% = \frac{2.33}{2.37} \approx 98.31 \% $$ 

Efektywność kodu Shannona-Fano
$$\eta_{\mathcal{D}} = \frac{H(\mathcal{S})}{L(\mathcal{D})}* 100\% = \frac{2.33}{2.63} \approx 88.59 \% $$ 

\end{document}



