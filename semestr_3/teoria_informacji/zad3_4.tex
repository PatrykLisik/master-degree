\documentclass[12pt]{article}
\usepackage[utf8]{inputenc}
\usepackage{polski}
\usepackage{enumitem}
\usepackage{amsmath}
\usepackage{tikz}
\usepackage{gauss}
\usepackage{xparse}

\ExplSyntaxOn
\keys_define:nn { gauss }
 {
  type .tl_set:N = \l_gauss_type_tl,
  type .initial:n = {},
  right .code:n = \tl_set:cn { g@post } { \relax$ },
  right .value_forbidden:n = true,
  spread .tl_set:N = \l_gauss_spread_tl,
  spread .initial:n = 1,
  colsep .dim_set:N = \l_gauss_colsep_dim,
  colsep .initial:n = \arraycolsep,
 }
\NewDocumentEnvironment{xgmatrix}{O{}}
 {
  \keys_set:nn { gauss } { #1 }
  \normalbaselines % <-----------  ADDED
  \linespread{\l_gauss_spread_tl}\selectfont
  \setlength{\arraycolsep}{\l_gauss_colsep_dim}
  \begin{gmatrix}[\l_gauss_type_tl]
 }
 {
  \end{gmatrix}
 }
\ExplSyntaxOff


\title{Zadanie 3--4}
\author{Patryk Lisik}
\date{\(11\) Luty  2024}

\begin{document}

\maketitle
\renewcommand{\abstractname}{Treść}

\begin{abstract}
    Niech $\mathcal{C}$ będzie $[6,3]$-birnarnym kodem liniowym rozpinanym przez bazowe słowa kodowe
    $\mathbf{u}_1 = 011011$, $\mathbf{u}_2 = 101101$ i $\mathbf{u}_3 = 111000$. Wyznacz macierz generującą $\mathbf{G}$
    dla kodu $\mathcal{C}$ w postaci systematycznej, następnie wyznacz macierz sprawdzanai parzystości $\mathbf{H}$ dla 
    kodu $\mathcal{C}$. Znajdź słowo kodowe $\mathbf{c}$ dla wiadomości $110$ i sprawdź, $\mathbf{cH^T}=0 $. Znajdź tempo
    $R$ i minimalną odległość $d$ dla kodu $\mathcal{C}$. Wyznacz tablę syndormu dla $\mathcal{C}$; jakie wzorce błędu 
    są przez niego poprawiane? Znajdź prawdopodobieństwo błędnego dekodowania $Pr_E$ przy przesyłaniu wiadomości 
    zadkodowanej tym kodem przez binarny kanał symertyczny $\Gamma \circ P < \frac{1}{2} $ i regule dekodowania $\Delta$ 
    przez najbiższgo sąsiada. 

\end{abstract}


\section*{Rozwiązanie}

\subsection*{Macierz generująca $ \mathbf{G} $ i macierz parzystości $\mathbf{H}$ }
$$ \mathbf{G'} = \begin{pmatrix}
    0 & 1 & 1 & 0 & 1 & 1 \\
    1 & 0 & 1 & 1 & 0 & 1 \\
    1 & 1 & 1 & 0 & 0 & 0 \\
\end{pmatrix}
$$ 

\begin{multline*}
\mathbf{G} = \begin{gmatrix}[b]
    0 & 1 & 1 & 0 & 1 & 1 \\
    1 & 0 & 1 & 1 & 0 & 1 \\
    1 & 1 & 1 & 0 & 0 & 0 
\rowops
 \swap{0}{1}
\end{gmatrix} = \begin{gmatrix}[b]
    1 & 0 & 1 & 1 & 0 & 1 \\
    0 & 1 & 1 & 0 & 1 & 1 \\
    1 & 1 & 1 & 0 & 0 & 0 
    \rowops
\add[]{0}{2}
\end{gmatrix} = 
 \begin{gmatrix}[b]
    1 & 0 & 1 & 1 & 0 & 1 \\
    0 & 1 & 1 & 0 & 1 & 1 \\
    0 & 1 & 0 & 1 & 0 & 1 
    \rowops
\add[]{1}{2}
\end{gmatrix} = \\
 = 
 \begin{gmatrix}[b]
    1 & 0 & 1 & 1 & 0 & 1 \\
    0 & 1 & 1 & 0 & 1 & 1 \\
    0 & 0 & 1 & 1 & 1 & 0 
    \rowops
\add[]{2}{0}
\add[]{2}{1}
\end{gmatrix} = 
 \begin{gmatrix}[b]
    1 & 0 & 0 & 0 & 1 & 1 \\
    0 & 1 & 0 & 1 & 0 & 1 \\
    0 & 0 & 1 & 1 & 1 & 0 
\end{gmatrix}
\end{multline*} 

$$\mathbf{H} = \begin{pmatrix}
    0 & 1 & 1 & 1 & 0 & 0 \\
    1 & 0 & 1 & 0 & 1 & 0 \\
    1 & 1 & 0 & 0 & 0 & 1
\end{pmatrix} $$

\subsection*{Obliczenie wiadomości i sprawdzenie poprawności} 

$$c=(110) \cdot \begin{pmatrix}
     1 & 0 & 0 & 0 & 1 & 1 \\
    0 & 1 & 0 & 1 & 0 & 1 \\
    0 & 0 & 1 & 1 & 1 & 0 
\end{pmatrix} = (110110)
$$

$$ c \mathbf{H}^T = (110110) \cdot \begin{pmatrix}
    0 & 1 & 1 \\
    1 & 0 & 1 \\
    1 & 1 & 0 \\
    1 & 0 & 0 \\
    0 & 1 & 0 \\
    0 & 0 & 1 \\
\end{pmatrix} = (000)
$$

\subsection*{Tempo transmisji kodu $R$}
$$R = \frac{k}{n} = \frac{3}{6} = \frac{1}{2} $$

\end{document}
