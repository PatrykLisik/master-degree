%%%%%%%%%%%% Attribution %%%%%%%%%%%%
% This template was created by 
% Chuck F. Rocca at WCSU and may be
% copied and used freely for 
% non-commercial purposes.
% 10-17-2021
%%%%%%%%%%%%%%%%%%%%%%%%%%%%%%%%%%%%%

%%%%%%% Start Document Header %%%%%%%
% In creating a new document
% copy and paste the header 
% as is.
%%%%%%%%%%%%%%%%%%%%%%%%%%%%%%%%%%%%%


\documentclass[12pt]{article}
\usepackage[utf8]{inputenc}
\usepackage{color}
\usepackage[utf8]{inputenc}
\usepackage{polski}
\usepackage[polish]{babel}
\usepackage{listings}
\usepackage{xcolor}
\usepackage{enumitem}
\usepackage{forest}
\usepackage{standalone}

\definecolor{codegreen}{rgb}{0,0.6,0}
\definecolor{codegray}{rgb}{0.5,0.5,0.5}
\definecolor{codepurple}{rgb}{0.58,0,0.82}
\definecolor{backcolour}{rgb}{0.95,0.95,0.92}
\lstdefinestyle{mystyle}{
  backgroundcolor=\color{backcolour},   commentstyle=\color{codegreen},
  keywordstyle=\color{magenta},
  numberstyle=\tiny\color{codegray},
  stringstyle=\color{codepurple},
  basicstyle=\ttfamily\footnotesize,
  breakatwhitespace=false,         
  breaklines=true,                 
  captionpos=b,                    
  keepspaces=true,                 
  numbers=left,                    
  numbersep=5pt,                  
  showspaces=false,                
  showstringspaces=false,
  showtabs=false,                  
  tabsize=2
}
%"mystyle" code listing set
\lstset{style=mystyle}

\definecolor{tab1}{gray}{0.94}
\definecolor{tab2}{rgb}{1, 1, 1}
\def\code#1{\texttt{#1}}

%%%% Header Information %%%%
    %%% Document Settings %%%%
    \usepackage[utf8]{inputenc}
    \usepackage[
        twoside,
        top=1in,
        bottom=0.75in,
        inner=0.5in,
        outer=0.5in
    ]{geometry}
    \pagestyle{myheadings}

%%%% Additional Commands to Load %%%%
    \usepackage{tcolorbox}
    \tcbuselibrary{skins}
    \usepackage{minted}
    \usepackage{color}
    \usepackage{tikz}
    \usetikzlibrary{calc}
    \usepackage{tabularx,colortbl}
    \usepackage{amsfonts,amsmath,amssymb}
    \usepackage{titling}
    \usepackage{mathrsfs}
    \usepackage{calc}

%%%% Commands to Define Homework Boxes %%%%
%%%% Box Definition %%%%
    \newtcolorbox{prob}[1]{
    % Set box style
        sidebyside,
        sidebyside align=top,
    % Dimensions and layout
        width=\textwidth,
        toptitle=2.5pt,
        bottomtitle=2.5pt,
        righthand width=0.20\textwidth,
    % Coloring
        colbacktitle=gray!30,
        coltitle=black,
        colback=white,
        colframe=black,
    % Title formatting
        title={
            #1 \hfill Grade:\hspace*{0.14\textwidth}
        },
        fonttitle=\large\bfseries
    }

%%%% Environment Definition %%%%
    \newenvironment{problem}[1]{
        \begin{prob}{#1}
    }
    {
        \tcblower
        \centering
        \textit{\scriptsize\bfseries Faculty Comments}
        \end{prob}
    }



%%%% Document Information %%%%
    \title{Zadanie 3}
    \author{Patryk Lisik}
    \date{\(13\) Styczeń  2023}

%%%%%%% End Document Header %%%%%%%


%%%% Begin Document %%%%
% note that the document starts with
% \begin{document} and ends with
% \end{document}
%%%%%%%%%%%%%%%%%%%%%%%%

\begin{document}

%%%% Format Running Header %%%%%
\markboth{\theauthor}{\thetitle}

%%%% Insert the Title Information %%%
\maketitle


%%%% General Description of the Document %%%%
\begin{abstract}
Programista ma napisać kod obliczający wartości funkcji $H(n)$ określonej następującymi równaniami:

\begin{align*}
&H(0)=1 \\
&H(1)=3 \\
&H(n) = [1 H(n-1)][2+H(n-2]
\end{align*}

Ponieważ jest on zafascynowany pomysłem rekurencji, to natychmiast napisał następujący kod:
\begin{lstlisting}[mathescape=true]
function H(n)
    if n $\le$ 1 then
        return 2$\cdot$n + 1
    else
        return (1+h(n-1))$\cdot$(2+h(n-2))
    end if
end function
\end{lstlisting}

Pomimo swej elegancji i prostoty, algorytm ten jest niewydajny
dla dużych wartości $n$
\begin{enumerate}[label=(\alph*)]
    \item Zbuduj drzewo wszystkiech wywołań funkcji podczas wykonywania wywołania $H(6)$ .
    Oblicz całkowitą liczbę wywołań funkcji dla $H(6)$.
    \item Niech $a_n$ będzie całkowitą liczba wywołań funkcji przy obliczaniu H(n).
    Napisz i rozwiąż związek rekurencyjny spełniany przez $a_n$.
    \item Przepisz kod dla $H(n)$ tak, aby był on iteracyjny i wydajny.
\end{enumerate}
\end{abstract}


%%%% Introduction to the General Template %%%%
\section*{a) Drzewo wywołań}
\begin{figure}[H]
    \includestandalone[width=\textwidth]{tree}
  \caption{Drzewo wywołań funkcji H}
  \label{fig:tikz:H}
\end{figure}

\section*{b) Związek rekurencyjny ilości wywołań}
Ilość wywołań można zdefiniować jako sumę ilości wywołań na poprzednich poziomach dodać obecny poziom. 
Jest to rekurencja liniowa niejednorodna drugiego rzędu. 
\begin{align*}
&a_0 = 1\\
&a_1 = 1 \\
&a_n = a_{n-2}+a_{n-1}+1
\label{eq:recursion}
\end{align*}

Rozwiążmy rekurencję liniową jednorodną $a_n^o = a_{n-2}+a_{n-1}$.
Równanie charakterystyczne
\begin{align*}
&r^2=r+1 \\
&r_1=\frac{1+\sqrt{5}}{2} \\
&r_2=\frac{1-\sqrt{5}}{2}  
\end{align*}

Rekurencja $a_n^o$ ma rozwiązanie postaci 
$$ 
a_n^o = \alpha_1r_1^n+\alpha_2r_2^n
$$

Znajdujemy szczególne rozwiązanie dla rekurencji niejednorodnej.

\begin{align*}
&b_n=b_{n-2}+b_{n-1}+1 \\
&b_n=\beta_0 \\
&\beta_0=\beta_0+\beta_0+1 \\
&\beta_0=-1 \\
\end{align*}

Sumujemy ogólne rozwiązanie rekurencji jednorodnej $a_n^o$ i rozwiązanie szczeólne części niejednorodnej.
$$
a_n= \alpha_1\left(\frac{1+\sqrt{5}}{2}\right)^n+\alpha_2\left(\frac{1-\sqrt{5}}{2}\right)^n - 1
$$

Podstawiamy znane stałe i obliczamy $\alpha_1$ i $\alpha_2$

$$
\begin{cases}
a_0=1 \\
a_1=1
\end{cases}
$$

$$
\begin{cases}
a_0 = \alpha_1r_1^0+\alpha_2r_2^0 - 1 = \alpha_1 + \alpha_2 - 1 \implies \alpha_2= -\alpha_1+2  \\
a_1 = \alpha_1r_1^1+\alpha_2r_2^1 - 1 = \alpha_1 \frac{1+\sqrt{5}}{2}+\alpha_2\frac{1-\sqrt{5}}{2} - 1
\end{cases}
$$

$$
1 = \alpha_1 \frac{1+\sqrt{5}}{2}+(-\alpha_1+2)\frac{1-\sqrt{5}}{2} - 1
$$
$$
2 = \alpha_1 \frac{1+\sqrt{5}}{2}+\alpha_1\frac{\sqrt{5}-1}{2}+1-\sqrt{5}
$$

$$
1 = \alpha_1 (\frac{1+\sqrt{5}}{2}+\frac{\sqrt{5}-1}{2})-\sqrt{5}
$$
$$
1 = \alpha_1\sqrt{5} -\sqrt{5}
$$
$$
\alpha_1 = \frac{1+\sqrt{5}}{\sqrt{5}} = \frac{5+\sqrt{5}}{5}
$$

$$
\begin{cases}
\alpha_1 = \frac{\sqrt{5}+5}{5} \\
\alpha_2 =  -\alpha_1+2 = \frac{5-\sqrt{5}}{5} 
\end{cases}
$$

Finalnie
$$
a_n= \frac{\sqrt{5}+5}{5}\left(\frac{1+\sqrt{5}}{2}\right)^n+\frac{5-\sqrt{5}}{5} \left(\frac{1-\sqrt{5}}{2}\right)^n - 1
$$

\section*{c) Implementacja iteracyjna}
Iteracyjna implementacja funkcji $H$ w języku Python 3
\begin{lstlisting}[language=python]
def H(x):
    if x == 0:
        return 1
    if x == 1:
        return 3
    h_0 = 1
    h_1 = 3
    h=None
    for _ in range(x-1):
        h = (1+h_1)*(2+h_0)
        h_0 = h_1
        h_1 = h
    return h
\end{lstlisting}
\end{document}
