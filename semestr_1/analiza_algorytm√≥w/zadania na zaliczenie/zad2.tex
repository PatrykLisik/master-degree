%%%%%%%%%%%% Attribution %%%%%%%%%%%%
% This template was created by 
% Chuck F. Rocca at WCSU and may be
% copied and used freely for 
% non-commercial purposes.
% 10-17-2021
%%%%%%%%%%%%%%%%%%%%%%%%%%%%%%%%%%%%%

%%%%%%% Start Document Header %%%%%%%
% In creating a new document
% copy and paste the header 
% as is.
%%%%%%%%%%%%%%%%%%%%%%%%%%%%%%%%%%%%%


\documentclass[12pt]{article}
\usepackage[utf8]{inputenc}
\usepackage{polski}
\usepackage{enumitem}


%%%% Header Information %%%%
    %%% Document Settings %%%%
    \usepackage[utf8]{inputenc}
    \usepackage[
        twoside,
        top=1in,
        bottom=0.75in,
        inner=0.5in,
        outer=0.5in
    ]{geometry}
    \pagestyle{myheadings}

%%%% Additional Commands to Load %%%%
    \usepackage{tcolorbox}
    \tcbuselibrary{skins}
    \usepackage{minted}
    \usepackage{color}
    \usepackage{tikz}
    \usetikzlibrary{calc}
    \usepackage{tabularx,colortbl}
    \usepackage{amsfonts,amsmath,amssymb}
    \usepackage{titling}
    \usepackage{mathrsfs}
    \usepackage{calc}

%%%% Commands to Define Homework Boxes %%%%
%%%% Box Definition %%%%
    \newtcolorbox{prob}[1]{
    % Set box style
        sidebyside,
        sidebyside align=top,
    % Dimensions and layout
        width=\textwidth,
        toptitle=2.5pt,
        bottomtitle=2.5pt,
        righthand width=0.20\textwidth,
    % Coloring
        colbacktitle=gray!30,
        coltitle=black,
        colback=white,
        colframe=black,
    % Title formatting
        title={
            #1 \hfill Grade:\hspace*{0.14\textwidth}
        },
        fonttitle=\large\bfseries
    }

%%%% Environment Definition %%%%
    \newenvironment{problem}[1]{
        \begin{prob}{#1}
    }
    {
        \tcblower
        \centering
        \textit{\scriptsize\bfseries Faculty Comments}
        \end{prob}
    }



%%%% Document Information %%%%
    \title{Zadanie 2}
    \author{Patryk Lisik}
    \date{\(17\) Styczeń  2023}

%%%%%%% End Document Header %%%%%%%


%%%% Begin Document %%%%
% note that the document starts with
% \begin{document} and ends with
% \end{document}
%%%%%%%%%%%%%%%%%%%%%%%%

\begin{document}

%%%% Format Running Header %%%%%
\markboth{\theauthor}{\thetitle}

%%%% Insert the Title Information %%%
\maketitle


%%%% General Description of the Document %%%%
\begin{abstract}
Poniższe funkcje są wymienione w przypad-kowym porządku. Uporządkuj je ze względu na ich tempowzrostu począwszy od funkcji najwolniej do najszybciej ro-snącej. Jeżeli dwie funkcje mają to samo tempo wzrostu (tzn. $f_1(n)=\Theta(f_2(n))$, o uporządkuje je na podstawie ich róż-nicy:
\begin{enumerate}[label=(\alph*)]
    \item $n$
    \item $\log_2n$
    \item $n \log_2n$
    \item $n!$
    \item $2^n$
    \item $n^2$
    \item $\log_2 n!$
    \item $n^2$
    \item $\sqrt{n}\log_2n$
    \item $n\sqrt{n}$
    \item $\left(\log_2n\right)^2$
    \item $\log_2 n^2$
    \item $\sqrt{n\log_2n}$
    \item $n\sqrt{\log_2n}$
    \item $1$
    \item $n(\log_2n)^2$
\end{enumerate}
\end{abstract}


%%%% Introduction to the General Template %%%%
\section{Funkcje uszeregowane po tempie wzrostu}
\begin{enumerate}
    \item o) $1$
    \item b) $\log_2n$
    \item h) $\sqrt{n}\log_2n$
    \item l) $\log_2 n^2$
    \item m) $n\sqrt{\log_2n}$
    \item k) $\left(\log_2n\right)^2$
    \item i) $\sqrt{n}\log_2n$
    \item a) $n$
    \item n) $n\sqrt{\log_2n}$
    \item f) $\log_2 n!$
    \item j) $n\sqrt{n}$
    \item c) $n \log_2n$
    \item g) $n^2$
    \item p) $n(\log_2n)^2$
    \item e) $2^n$
    \item d) $n!$
\end{enumerate}

\end{document}
