%%%%%%%%%%%% Attribution %%%%%%%%%%%%
% This template was created by 
% Chuck F. Rocca at WCSU and may be
% copied and used freely for 
% non-commercial purposes.
% 10-17-2021
%%%%%%%%%%%%%%%%%%%%%%%%%%%%%%%%%%%%%

\documentclass[12pt]{article}
\usepackage[utf8]{inputenc}
\usepackage{amssymb}
\usepackage{polski}
\usepackage[polish]{babel}
\usepackage{listings}
\usepackage{url}
\usepackage{xcolor}

\definecolor{codegreen}{rgb}{0,0.6,0}
\definecolor{codegray}{rgb}{0.5,0.5,0.5}
\definecolor{codepurple}{rgb}{0.58,0,0.82}
\definecolor{backcolour}{rgb}{0.95,0.95,0.92}
\lstdefinestyle{mystyle}{
  backgroundcolor=\color{backcolour},   commentstyle=\color{codegreen},
  keywordstyle=\color{magenta},
  numberstyle=\tiny\color{codegray},
  stringstyle=\color{codepurple},
  basicstyle=\ttfamily\footnotesize,
  breakatwhitespace=false,         
  breaklines=true,                 
  captionpos=b,                    
  keepspaces=true,                 
  numbers=left,                    
  numbersep=5pt,                  
  showspaces=false,                
  showstringspaces=false,
  showtabs=false,                  
  tabsize=2
}
%"mystyle" code listing set
\lstset{style=mystyle}

\definecolor{tab1}{gray}{0.94}
\definecolor{tab2}{rgb}{1, 1, 1}
\def\code#1{\texttt{#1}}

%%%% Header Information %%%%
    \include{header}

%%%% Document Information %%%%
    \title{Zadanie 1}
    \author{Patryk Lisik}
    \date{\(10\) Styczeń  2023}

%%%%%%% End Document Header %%%%%%%


%%%% Begin Document %%%%
% note that the document starts with
% \begin{document} and ends with
% \end{document}
%%%%%%%%%%%%%%%%%%%%%%%%

\begin{document}

%%%% Format Running Header %%%%%
\markboth{\theauthor}{\thetitle}

%%%% Insert the Title Information %%%
\maketitle


%%%% General Description of the Document %%%%
\begin{abstract}
Mędrzec uczynił sułtanowi znaczną przysługę,za którą sułtan postanowił go wynagrodzić według jego życzenia. Mędrzec odpowiedział: „Nie żądam wiele. Wystarczy, że położysz jedno ziarno pszenicy na pierwszym polu szachow-nicy, dwa ziarna na drugim polu, dwa razy po dwa ziarna na kolejnym polu i tak dalej, podwajając za każdym razem liczbę ziaren, aż zapełnisz wszystkie pola na szachownicy”. 
Jak wiele ziaren musi zebrać sułtan, aby zapłacić mędrcowi za przysługę?  Podaj rozwiązanie w zwartej postaci.
Jaki typ zmiennej dla tej wartości musi zadeklarować nadworny informatyk sułtana, aby mieć pewność, że poprawnie napisał w języku C++ program obliczający liczbę ziaren pszenicy dla mędrca?
\end{abstract}


%%%% Introduction to the General Template %%%%
\section{Obliczenie ilości ziaren}
Rozpatrzmy standardową szachownicę $8\times8$, co oznacza 64 pola. 
Ilość ziaren na każdym polu można wyrazić jako potęgę liczby 2. 
Na pierwszym polu znajduje się jedno ziarnko ($2^0$), na drugim polu są dwa ziarnka($2^1$), na n-tym polu znajduje się $2^{n-1}$ ziarenek.
Zakładając przybliżenie $2^{10}=10^3$, ich sumę można wyrazić jako. 
$$
n^0+n^1+\dots+n^{63}=\sum^{63}_{n=0} 2_n = 2^{64} = \left(2^{10}\right)^6 \cdot 2^4 \approx 16 \cdot 10^{18}
$$
\section{Typ zmiennej w C++}
Nawet bez sumowania łatwo zauważyć analogię do liczb binarnych. 
Każde pole szachownicy oznacza jeden bit w zmiennej bez znaku. 
Długość zmiennej w języku c++ jest zależna od platformy. 
Na platformie \code{LP64} wystarczy zmienna typu \code{unsigned long int}. 
Na wszystkich platformach można wykorzystać typ \code{unsigned long long int}.
Krótszą bardziej czytelną formę, niezależną od platformy można znaleźć w nagłówku \code{<cstdint>} jako \code{uint64\_t}.
Wykonanie poniższego kodu można znaleźć tutaj \url{https://ideone.com/tMhW9w}.
Test wykonano na kompilatorze gcc 8.3 na środowisku 64-bit. 

\begin{lstlisting}[language=C++ ,caption= Kod obliczający ilość ziarenek.]
#include <stdint.h>
#include <cstdint>
#include <iostream>
 
int main() {
	long unsigned int seed_count = 0;
	long unsigned int current_count = 1;
	for(int i=0;i<64;i++){
		seed_count+=current_count;
		current_count*=2;
	}
	std::cout<<"Seeds count: "<<seed_count<<"\n";
	return 0;
}

\end{lstlisting}


\end{document}
