%%%%%%%%%%%% Attribution %%%%%%%%%%%%
% This template was created by 
% Chuck F. Rocca at WCSU and may be
% copied and used freely for 
% non-commercial purposes.
% 10-17-2021
%%%%%%%%%%%%%%%%%%%%%%%%%%%%%%%%%%%%%

\documentclass[12pt]{article}
\usepackage[utf8]{inputenc}
\usepackage{color}
\usepackage[utf8]{inputenc}
\usepackage{amssymb}
\usepackage{polski}
\usepackage[polish]{babel}
\usepackage{amsmath}
\usepackage{amsfonts}
\usepackage{lastpage}
\usepackage{listings}
\usepackage{url}
\usepackage{xcolor}
% \usepackage{enumitem}

\definecolor{codegreen}{rgb}{0,0.6,0}
\definecolor{codegray}{rgb}{0.5,0.5,0.5}
\definecolor{codepurple}{rgb}{0.58,0,0.82}
\definecolor{backcolour}{rgb}{0.95,0.95,0.92}
\lstdefinestyle{mystyle}{
  backgroundcolor=\color{backcolour},   commentstyle=\color{codegreen},
  keywordstyle=\color{magenta},
  numberstyle=\tiny\color{codegray},
  stringstyle=\color{codepurple},
  basicstyle=\ttfamily\footnotesize,
  breakatwhitespace=false,         
  breaklines=true,                 
  captionpos=b,                    
  keepspaces=true,                 
  numbers=left,                    
  numbersep=5pt,                  
  showspaces=false,                
  showstringspaces=false,
  showtabs=false,                  
  tabsize=2
}
 \newcommand{\floor}[1]{\left\lfloor #1  \right\rfloor}
 \newcommand{\ceil}[1]{\left\lceil #1 \right\rceil} 

%"mystyle" code listing set
\lstset{style=mystyle}

\definecolor{tab1}{gray}{0.94}
\definecolor{tab2}{rgb}{1, 1, 1}
\def\code#1{\texttt{#1}}

%%%% Header Information %%%%
    \include{header}

%%%% Document Information %%%%
    \title{Zadanie 4}
    \author{Patryk Lisik}
    \date{\(20\) Styczeń  2023}


\begin{document}

%%%% Format Running Header %%%%%
\markboth{\theauthor}{\thetitle}

%%%% Insert the Title Information %%%
\maketitle
\newtheorem{lemma}{Lemat}

%%%% General Description of the Document %%%%
\begin{abstract}
Każdą dodatnią liczbę całkowitą $n$ można jednoznacznie wyrazić w postaci dwójkowej: $n=b_k2_k+b_{k-1}^{k-1}\dotsb_12^1+b_02^0$, gdzie $b_i \in {0,1}$ i $b_k=1$ dla $n>0$.
Wskaźnik najbardziej znaczącego bitu $b_k$ jest równy $k=\floor{\log_2 n}$ więc całkowita liczba bitów w rozwinięciu dwójkowym liczby $n>0$ wynosi $\ell_n =\floor{log_2n}+1$.
% najbardziej znaczązy bit - bit o największej warotości 
Niech $s_n$ będzie liczbą '1' w w zapisie dwójkowym liczby $n$ (tj. $s_n = \sum^k_{i=0}b_k $).
Łatwo sprawdzić, że $s_0=0 ,s_1=0, s_2=1, s_3=2, s_4=1,s_5=2,s_6=2, s_7=3, s_8=1$ itd. Ponieważ 
$$ \floor{\frac{n}{2}}=b_k2^{k-1}+b_{k-1}2^{k-2}+\dots+b_12^0$$
to wynika stąd, że liczba „1” w zapisie dwójkowym $\floor{\frac{n}{2}}$ jest taka sama, jak w zapisie dwójkowym $n$ (gdy $b_0=0$ lub jest o jeden mniejsza(gdy $b_0=0$).
Zauważ, ze $\forall_{n\ge 0} b_0 = n-2\floor{\frac{n}{2}}$.
Ostatecznie otrzymujemy związek rekurencyjny.
\begin{align*}
    &s_0=0, \\
    &s_n = s_{\floor{\frac{n}{2}}} + n - 2\floor{\frac{n}{2}}
\end{align*}

Znajdź wyrażenie dla $s_n$ zawierające co najwyżej jedną
sumę częściową pewnego ciągu oraz górne i dolne ograniczenie na tempo wzrostu $s_n$.


\end{abstract}



\begin{lemma}

\begin{align*}
    &\textbf{Rekurencja dla n} \ge \textbf{2 : }  a_n = ba_{\floor{\frac{n}{2}}}+n \\
    &\textbf{Rozwiązanie ogólne części jednorodnej: }  a^o_n=\alpha b^{\floor{\log n}}=\Theta(n^{\log b}) \\
    & \textbf{Rozwiązanie szczególne części niejednorodnej: }  a^s_n=\sum^{\floor{\log n}}_{k=0} \floor{\frac{n}{2^k}}b^k
\end{align*}

\end{lemma}

\begin{lemma}

\begin{align*}
    &\textbf{Rekurencja dla n} \ge \textbf{2 : }  a_n = ba_{\floor{\frac{n}{2}}}+\floor{\frac{n}{2}} \\
    &\textbf{Rozwiązanie ogólne części jednorodnej: }  a^o_n=\alpha b^{\floor{\log n}}=\Theta(n^{\log b}) \\
    & \textbf{Rozwiązanie szczególne części niejednorodnej: }  a^s_n=\sum^{\floor{\log n}}_{k=1} \floor{\frac{n}{2^k}}b^{k-1}
\end{align*}

\end{lemma}

\begin{lemma}[Twierdzenie o superpozycji]
Jeżeli $a_n=f_n$ i $a_n=g_n$ są odpowiednio rozwiązaniami rekurencji liniowych niejednorodnych:
\begin{align*}
    & a_n = c_1(n)a_{n-1}+c_2(n)a_{n-2}+\dots+c_n(n)a_0+\phi(n) \\
    & a_n = c_1(n)a_{n-1}+c_2(n)a_{n-2}+\dots+c_n(n)a_0+\psi(n) \\
\end{align*}
to $a_n = \alpha f_n + \beta g_n$ jest rozwiązaniem rekurencji liniowej niejednorodnej postaci:
$$ a_n = c_1(n)a_{n-1}+c_2(n)a_{n-2}+\dots+c_n(n)a_0+\alpha\phi(n)+\beta\psi(n) $$
\end{lemma}

\begin{lemma}
    $ n=\floor{\frac{n}{2}}+\ceil{\frac{n}{2}} $ 
\end{lemma}

\begin{lemma}[Z tabeli III]
    Rekurencja: $a_n = b_{\floor{n/2}} + \Theta(1)$ ma tempo wzrostu 
    $$
    \begin{cases}
    a_n = \Theta(1) &  0<b<1 \\
    a_n = \Theta(\log n) & b=1 \\
    a_n = \Theta(n^{\log b} & b>1 i a_1>\frac{1}{1-b}
    \end{cases}
    $$

    
\end{lemma}

\section*{Rozwiązanie rekurencji $s_n$}

Rozważmy ciągi 
\begin{align*}
&c_n = c_{\floor{\frac{n}{2}}}+\floor{\frac{n}{2}} \\
&d_n = d_{\floor{\frac{n}{2}}}+n \\
\end{align*}
o rozwiązaniach: 
\begin{align*}
    &c_n = \alpha 1^{\floor{\log n}} +  \sum^{\floor{\log n}}_{k=0} \floor{\frac{n}{2^k}}1^k= \alpha + \sum^{\floor{\log n}}_{k=0} \floor{\frac{n}{2^k}} \\
    &d_n = \alpha 1^{\floor{\log n}} +  \sum^{\floor{\log n}}_{k=1} \floor{\frac{n}{2^k}}1^{k-1}= \alpha + \sum^{\floor{\log n}}_{k=1} \floor{\frac{n}{2^k}} \\
\end{align*}
z twierdzenia o superpozycji: 
\begin{align*}
    s_n = \alpha + \sum^{\floor{\log n}}_{k=0} \floor{\frac{n}{2^k}} - 2 \left(\sum^{\floor{\log n}}_{k=1} \floor{\frac{n}{2^k}} \right) = \alpha +n - \sum^{\floor{\log n}}_{k=1} \floor{\frac{n}{2^k}}
\end{align*}

Lematy 1 i 2 są aplikowalne tylko dla $n\ge 2$ dlatego $\alpha$ musimy obliczyć dla współczynnika przynajmniej 2.
Manualnie obliczamy że $S_2 = 1$ i natępnie obliczamy $\alpha$.
$$
1=\alpha +2 - \sum^{\floor{\log 1}}_{k=1} \floor{\frac{2}{2^k}} \implies \alpha=-1
$$
Finalnie 
$$
s_n =  n - 1 - \sum^{\floor{\log n}}_{k=1} \floor{\frac{n}{2^k}}
$$
\section*{Górne i dolne ograniczenie $s_n$}
Podstawiając lemat 4 można wyrazić jako ciąg $s_n$ można wyrazić jako:
$$
s_n = s_{\floor{\frac{n}{2}}} + n - 2\floor{\frac{n}{2}} = s_{\floor{\frac{n}{2}}} + \ceil{\frac{n}{2}} -\floor{\frac{n}{2}} = s_{\floor{\frac{n}{2}}} + \Theta(1)
$$
Z lematu 5 wiemy, że 
$$ s_n = \Theta(\log n)$$

\end{document}
